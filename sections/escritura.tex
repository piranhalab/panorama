% aclarar algunas cuestiones sobre los módulos para que no sean tan críptico 

\section*{Diseño y escritura} % Aquí puede ir lo que escribió Dorian

El diseño de \textit{Panorama} consideró:

\begin{itemize}
\item Escritura de una experiencia ligera: tiempo de carga, percepción de movimiento del punto de vista de la cámara y percepción de otros usuarios en un espacio tridimensional.  
\item Utilización sencilla, adaptable a distintos dispositivos y a la capacidad técnica de los usuarios.
\item Transmisión audiovisual en vivo que permitiera simular la experiencia de asistencia a conciertos para eventos recurrentes y no recurrentes. 
\item Autonomía en el uso de recursos: el sistema no implicó un registro y buscó evadir el análisis del \textit{streaming} o de la conexión entre módulos para evitar transmisiones silenciadas por el uso de materiales con derechos de autor. 
\item Sistemas fáciles de instalar, recrear y escalar.
\item Tecnologías normalizadas, no privativas, con una marcada preferencia por el software libre.
\item Generación de experiencias personalizadas que respondiera a las necesidades performativas, posibilidades futuras abiertas y adaptables. 
\end{itemize}

\textit{Panorama} se escribió para entornos web, no requiere instalación de software adicional, más allá de un navegador web actual. Los avances en la tecnología web permiten la homogeneización de la experiencia y pueden abarcar un gran público. El sistema utiliza WebGL\footnote{\url{https://www.khronos.org/webgl/}(Consultado el \today)} para la renderización de imagen en el navegador. Este entorno permite la generación de espacios virtuales y tridimensionales donde se pueden alojar experiencias visuales, auditivas y presenciales. Javascript es el lenguaje de programación principal del proyecto debido a su uso normalizado en navegadores y aplicaciones web. Para fines prácticos, este sistema permite conectar las partes del sistema entero.

Three.js como \textit{framework} implementa webGL y asegura la rápida codificación y reutilización de código. Además, es posible importar modelos hechos en programas de modelado tridimensional, efectos y módulos para la reproducción de audio en web con Web Audio API\footnote{``La API de Audio Web provee un sistema poderoso y versatil para controlar audio en la Web, permitiendo a los desarrolladores escoger fuentes de audio, agregar efectos al audio, crear visualizaciones de audios, aplicar efectos espaciales (como paneo) y mucho más.'' \url{https://developer.mozilla.org/es/docs/Web/API/Web_Audio_API} (Consultado el \today)} y conceptos convencionales y compartidos de materiales y geometrías relacionados a programas de modelado en tes dimensiones. Como un punto adicional, este marco de trabajo cuenta con documentación actualizada y constante. La implementación de WebAssembly\footnote{``WebAssembly es un nuevo tipo de código que puede ser ejecutado en navegadores modernos — es un lenguaje de bajo nivel, similar al lenguaje ensamblador, con un formato binario compacto". \url{https://developer.mozilla.org/es/docs/WebAssembly} (Consultado el \today) } ha permitido la generación de sistemas rápidos y eficientes, sin capas de abstracción de software innecesarias.

El diseño del módulo de \textit{streaming} presentó una discusión sobre lo legal, la reproducción de audio y video en la web y uso de recursos. Las restricciones que imponen los servidores privados al contenido remixeado fue una de las motivaciones para implementar un sistema personalizado. De esta manera, el sistema evita el mercado exclusivo de servicios de streaming privados, con licencias de paga o limitados. De manera complementaria, el sistema de \textit{streaming} personalizado permite controlar la cantidad de recursos usados y usuarios conectados. Los formatos para la transmisión de video en web explorados fueron: RTMP (Real Time Messaging Protocol), FLV (Flash Video), HLS (HTTP Live Streaming), video a través de WebSockets\footnote{``WebSockets es una tecnología avanzada que hace posible abrir una sesión de comunicación interactiva entre el navegador del usuario y un servidor.'' \url{https://developer.mozilla.org/es/docs/Web/API/WebSockets_API}(Consultado el \today)}, WebRTC (Web Real-Time Communication) y MPEG-DASH (Dynamic Adaptive Streaming over HTTP).

Las características actuales presentes en HTML5 (Quinta versión de HyperText Markup Language) permiten utilizar formatos de streaming no nativos con \textit{frameworks} de decodificación para el despliegue de video en formatos soportados por el navegador. \textit{Panorama} utiliza RTMP debido a 1) la robustez del protocolo, 2) la rapidez de transmisión de datos, 3) la documentación que existe sobre NGINX\footnote{``Nginx es un servidor web/proxy inverso ligero de alto rendimiento". \url{https://es.wikipedia.org/wiki/Nginx}(Consultado el \today)} RTMP y 4) el uso de FFmpeg\footnote{``FFmpeg es el marco de trabajo multimedia líder capaz de decodificar, codificar, transcodificar, mux, demux, transmitir, filtrar y reproducir casi cualquier cosa que los humanos o las máquinas hayan creado". \url{https://www.ffmpeg.org/about.html}(Consultado el \today)} para la recodificación y redimensión de video en la web. La plataforma utiliza FLV debido a la velocidad del protocolo (no se descarta el uso de otros protocolos en el futuro). Para el uso de FLV en el navegador, se utilizó FLV.js. 

Para la interacción entre usuarios se utilizó WebSockets. Esto implicó: 1) uso de chat, 2) compartición en tiempo real de eventos gestuales (rotación y posición) del avatar utilizado y 3) personalización de modelo, textura y nombre de cada usuario. 

Del lado del servidor  se usaron balanceadores de carga junto con modelos de configuración para la instalación y eliminación de recursos, haciendo independientes los servidores dedicados al \textit{streaming} y a servicios web. Esto permite que los espacios puedan ser escritos, eliminados o mantenidos según el diseño del evento en cuestión.

% Escribir sobre las peculiaridades del git que estamos referenciando 

