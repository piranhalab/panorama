% aclarar algunas cuestiones sobre los módulos para que no sean tan críptico 

\section*{Diseño y escritura} % Aquí puede ir lo que escribió Dorian

\color{BlueGreen}

% La presente investigación estuvo relacionada con la gamificación emergente\footnote{O Ludificación, del inglés \textit{gamification}. Para \cite{gamificacion} existe la gamificación intencional y emergente. El presente artículo hace referencia a la segunda: ``la gamificación se puede definir como un proceso cultural gradual y emergente, aunque no intencional, derivada del compromiso cada vez más generalizado con los juegos e interacciones lúdicas."} tácita en la escritura de proyectos como \textit{Panorama}. Esto implica 1) tecnología, por ejemplo pantallas, combinaciones de teclas para la exploración de espacios pero también dispositivos de realidad virtual, 2) diseño visual y sonoro de los espacios, objetos digitales en espacios tridimensionales, audio inmersivo, 3) intercambio de información para la co-presencia en tiempo real dentro de espacios digitales o el multijugador e incluso 4) narrativas e imaginarios convenidos, explorados desde el diseño de la experiencia y esperados del lado del público. 

\color{black}

El diseño de \textit{Panorama} tuvo en cuenta las siguientes características:

\begin{itemize}
\item Escritura de una experiencia ligera. Reducción de tiempo de carga, percepción de movimiento del punto de vista de la cámara y de otros usuarios en un espacio tridimensional. 
\item Uso sencillo, adaptable del sitio a distintos dispositivos y a la capacidad técnica de usuarios no especializados. 
\item Transmisión audiovisual en vivo para generar la experiencia de público asistente a conciertos. 
\item Autonomía en el uso de recursos. El sistema no implicó un registro y buscó evadir el análisis de la transmisión por parte de terceros para evitar transmisiones silenciadas por el uso de materiales restringidos. 
\item Sistemas fáciles de instalar, recrear y escalar. 
Tecnologías normalizadas, no privativas, con una marcada preferencia por el software libre. 
\item Generación de experiencias personalizadas que respondieron a las necesidades performativas de los artistas con posibilidades futuras abiertas y adaptables
 
\end{itemize}

% Esto primero puede tener que ver con la propuesta de David Ogborn que está por ahí sobre estuary y la cero instalación 

Panorama funciona en navegadores web actuales como Chrome, Chromium o Firefox, y no requiere instalación de software adicional. En este sentido el proyecto persiguió que la experiencia fuera compartida y explorada por un gran público.  

%\deleted[id=EO, comment={Versión del director}]{Para asegurar la rápida codificación y la reutilización de código, \textit{Panorama} utiliza Three.js como \textit{framework} principal. Implementa WebGL para la renderización de imagen en el navegador y la importación de objetos 3d previamente modelados y permite el uso de convenciones como materiales y geometrías. Para la reproducción de muestras de audio en Web, three.js incorpora Web Audio API. La implementación de WebAssemblyha permitido la generación de sistemas rápidos y eficientes, sin capas de abstracción de software innecesarias.} % No encuentro la conexión de lo anterior con esto

 % \textit{Panorama} utiliza three.js debido a que cuenta con documentación actualizada y constante. % Quité esto

%\color{BlueGreen}
  
% Three.js fue el entorno principal para el diseño del escenario y el despliegue de gráficos. % Esto quedó comentado por sugerencia de Hernani

% De manera secundaria, el proyecto utilizó este marco de trabajo debido a que cuenta con documentación actualizada y constante. % Esto queda comentado 

Si bien, hay distintos conceptos que nos permiten hablar del diseño y arquitectura de espacios virtuales, el presente escrito reactiva el término \emph{ciberespacio}\footnote{"El término \emph{ciberespacio} puede ser usado vagamente para describir la interconexión espacial e invisible de computadoras en el Internet y es aplicado a casi cualquier experiencia espacial y virtual creada en una computadora`` \citep{cyberspace}} como un punto de partida para describir la actividad relacionada con \textit{Panorama}. Consideramos que la relación expresada en Panorama, retomada del ciberespacio como una noción narrativa y de inmersividad como una forma de resolución tecnológica de exploración de una cámara en un espacio tridimensional, con audio posicionado,\footnote{``En \textit{Web Audio}, espacializaciones complejas en 3D son creadas utilizando PannerNode, que en términos legos básicamente son muchas matemáticas \textit{cool} para hacer que el audio aparezca en un espacio 3D. Piensa en el audio volando sobre de ti, trepando atras de ti, moviéndose enfrente de ti" (\url{https://developer.mozilla.org/en-US/docs/Web/API/Web_Audio_API/Web_audio_spatialization_basics)} (Consultado el \today)}, puede arrojar algunas reflexiones, posibilidades y posicionamientos implicados en la escritura de Panorama. % Falta desarrollar esto 
\color{black}

El diseño del módulo de \emph{streaming} presentó una discusión sobre el uso de materiales restringidos por copyright, la reproducción de audio y video en la web y el uso de recursos. Las restricciones que imponen plataformas con servidores privados como YouTube o Facebook Live al contenido remixeado fue una de las motivaciones para implementar un sistema personalizado. Con esto, el sistema evita el mercado exclusivo de servicios de streaming privados, con licencias de paga o limitadas. También permite controlar la cantidad de recursos usados y usuarios conectados. 

%\deleted[id=EO, comment={Corte del director}]{Los formatos para la transmisión de video en web explorados fueron: RTMP (Real Time Messaging Protocol), FLV (Flash Video), HLS (HTTP Live Streaming), video a través de WebSockets, WebRTC (Web Real-Time Communication) y MPEG-DASH (Dynamic Adaptive Streaming over HTTP).}% Queda pendiente ver si esto se queda o se va 

%\deleted[id=EO, comment={Corte del director}]{Las características actuales presentes en HTML5 permiten utilizar formatos de streaming no nativos con \textit{frameworks} de decodificación para el despliegue de video en formatos soportados por el navegador. \textit{Panorama} utiliza RTMP debido a 1) la robustez del protocolo, 2) la rapidez de transmisión de datos, 3) la documentación que existe sobre NGINX RTMP y 4) el uso de FFmpeg para la recodificación y redimensión de video en la web. La plataforma utiliza FLV debido a la velocidad del protocolo. Para el uso de FLV en el navegador, se utilizó FLV.js.} 

% \deleted[id=EO, comment={El corte del director}]{Para la interacción entre usuarios el sistema utilizó WebSockets. Esto implicó: 1) uso de un chat, 2) compartición en tiempo real de eventos gestuales (rotación y posición) del avatar utilizado y 3) personalización de modelo, textura y nombre de cada usuario.} 

Los espacios escritos consistieron en incipientes modos de exploración copresencial que guardaban una relación entre la posición y la rotación de la cámara, controlada por medio del ratón y el teclado de la computadora. De manera complementaria, el espacio contaba con un chat. Para el control se utilizaron las teclas WASD y flechas de navegación siguiendo la convención de la cámara de videojuegos en primera persona. La información de posición y rotación fueron compartidas y compartieron la visualización del movimiento de avatares que representaron a los asistentes en el espacio digital. 

Nos detenemos en la exploración co-presencial para relacionar la interacción del público con la gamificación tácita en la escritura de proyectos como Panorama. Esto implica:

\begin{itemize}
\item Pantallas virtuales que despliegan la parte visual del concierto y combinaciones de teclas para la exploración de espacios.
\item Diseño visual y sonoro de los espacios, objetos digitales en espacios tridimensionales, audio inmersivo. 
Intercambio de datos para la copresencia en tiempo real de los usuarios dentro de espacios digitales. 
\item Narrativas e imaginarios convenidos, cinematográficos o literarios, explorados desde el diseño de la experiencia y esperados del lado del público. 
\end{itemize}

% Escribir sobre las peculiaridades del git que estamos referenciando 

% Cuerpo digital 

La concatenación de software estuvo delimitada por la experiencia de tres tipos de usuarios. Como la propuesta partió de eventos performáticos tipo concierto, la figura de intérpretes/artistas fue central. El mantenimiento del flujo performático expresado a partir de una transmisión de audio y video delimitó el aspecto técnico y estético del espacio. Sobre este punto, el objetivo principal fue que el streaming pudiera ser estable, eficiente y compatible con la convención de transmisión personalizada ejecutada con programas de interfaz gráfica como OBS (Open Broadcaster Software) o butt (broadcast using this tool), y herramientas que pudieran ejecutarse en línea de comandos como FFmpeg. El segundo tipo de agente fue el público que navegaba en el espacio y que realizaba gestos corporales virtuales sencillos: posición y rotación. Un tercer tipo de agente se involucró con los aspectos fuera del escenario, la logística de la transmisión y el mantenimiento del sistema en tiempo real.  

% 1) , 2)  3)  e incluso 4) narrativas e imaginarios convenidos, explorados desde el diseño de la experiencia y esperados del lado del público. 

Finalmente, del lado del servidor se usaron balanceadores de carga junto con modelos de configuración para la instalación y eliminación de recursos, haciendo independientes los servidores dedicados al \emph{streaming} y a servicios web. Esto permite que los espacios puedan ser escritos, eliminados o conservados a voluntad.  
