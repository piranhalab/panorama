% aclarar algunas cuestiones sobre los módulos para que no sean tan críptico 

\section*{Diseño y escritura} % Aquí puede ir lo que escribió Dorian

\color{BlueGreen}

% La presente investigación estuvo relacionada con la gamificación emergente\footnote{O Ludificación, del inglés \textit{gamification}. Para \cite{gamificacion} existe la gamificación intencional y emergente. El presente artículo hace referencia a la segunda: ``la gamificación se puede definir como un proceso cultural gradual y emergente, aunque no intencional, derivada del compromiso cada vez más generalizado con los juegos e interacciones lúdicas."} tácita en la escritura de proyectos como \textit{Panorama}. Esto implica 1) tecnología, por ejemplo pantallas, combinaciones de teclas para la exploración de espacios pero también dispositivos de realidad virtual, 2) diseño visual y sonoro de los espacios, objetos digitales en espacios tridimensionales, audio inmersivo, 3) intercambio de información para la co-presencia en tiempo real dentro de espacios digitales o el multijugador e incluso 4) narrativas e imaginarios convenidos, explorados desde el diseño de la experiencia y esperados del lado del público. 

\color{black}

Para el diseño de \textit{Panorama} tuvimos en cuenta las siguientes características:

\begin{itemize}
\item Escritura de una experiencia ligera: tiempo de carga, percepción de movimiento del punto de vista de la cámara y percepción de otros usuarios en un espacio tridimensional.  
\item Utilización sencilla, adaptable a distintos dispositivos y a la capacidad técnica de los usuarios.
\item Transmisión audiovisual en vivo que permitiera simular la experiencia de asistencia a conciertos para eventos recurrentes y no recurrentes. 
\item Autonomía en el uso de recursos: el sistema no implicó un registro y buscó evadir el análisis del \textit{streaming} o de la conexión entre módulos para evitar transmisiones silenciadas por el uso de materiales con derechos de autor. 
\item Sistemas fáciles de instalar, recrear y escalar.
\item Tecnologías normalizadas, no privativas, con una marcada preferencia por el software libre.
\item Generación de experiencias personalizadas que respondiera a las necesidades performativas, posibilidades futuras abiertas y adaptables. 
\end{itemize}

% Esto primero puede tener que ver con la propuesta de David Ogborn que está por ahí sobre estuary y la cero instalación 

\textit{Panorama} funciona en navegadores web actuales y no requiere instalación de software adicional. \textit{Javascript} es el lenguaje de programación principal del proyecto debido al uso normalizado en navegadores y aplicaciones web. De esta manera la experiencia puede ser compartida y puede abarcar un gran público. Para asegurar la rápida codificación y la reutilización de código, \textit{Panorama} utliza three.js como \textit{framework} principal. Implementa WebGL\footnote{\url{https://www.khronos.org/webgl/}(Consultado el \today)} para la renderización de imagen en el navegador y la importación de objetos 3d previamente modelados y permite el uso de convenciones para el trabajo en entornos tridimensionales como materiales y geometrías. Para la reproducción de muestras de audio en Web, three.js incorpora Web Audio API\footnote{``La API de Audio Web provee un sistema poderoso y versatil para controlar audio en la Web, permitiendo a los desarrolladores escoger fuentes de audio, agregar efectos al audio, crear visualizaciones de audios, aplicar efectos espaciales (como paneo) y mucho más.'' \url{https://developer.mozilla.org/es/docs/Web/API/Web_Audio_API} (Consultado el \today)}. \textit{Panorama} utiliza three.js debido a que cuenta con documentación actualizada y constante. La implementación de WebAssembly\footnote{``WebAssembly es un nuevo tipo de código que puede ser ejecutado en navegadores modernos — es un lenguaje de bajo nivel, similar al lenguaje ensamblador, con un formato binario compacto". \url{https://developer.mozilla.org/es/docs/WebAssembly} (Consultado el \today) } ha permitido la generación de sistemas rápidos y eficientes, sin capas de abstracción de software innecesarias. % No encuentro la conexión de lo anterior con esto

\color{BlueGreen}
  
Three.js permitió el diseño espacial de los escenarios creados específicamente para el ciclo de Conciertos EDGES. Si bien, hay distintos conceptos que nos permiten hablar del diseño y arquitectura de espacios virtuales, el presente escrito reactiva el término cyberespacio\footnote{``El término cyberespacio puede ser usado vagamente para describir la interconexión espacial e invisible de computadoras en el Internet y es aplicado a casi cualquier experiencia espacial y virtual creada en una computadora'' \citep{cyberspace}} como un punto de partida para describir la actividad relacionada con \textit{Panorama}. Puede ser contrapuesto con diversos conceptos que coinciden en la práctica, en la arquitectura de intercambio de información, en la denominación conceptual y en la convergencia de tecnologías expresamente mercantilizadas para la distribución de experiencias de Realidad Virtual como Oculus Rift. Realidad virtual también puede ser un concepto inicial, socialmente convenido para describir la experiencia de \textit{Panorama} pero para evitar contradicciones en la conducción de una experiencia en el cyberespacio hacia exploraciones hacia dispositivos/interfaces-mercancía específicas para la interacción en estos espacios, el presente artículo decide retomar la noción de inmersividad. Consideramos que la relación entre cyberespacio como una noción narrativa y de inmersividad como una forma de resolución técnica de exploración de una cámara en un espacio tridimensional, con audio posicionado\footnote{``En \textit{Web Audio}, espacializaciones complejas en 3D son creadas utilizando PannerNode, que en términos legos básicamente son muchas matemáticas \textit{cool} para hacer que el audio aparezca en un espacio 3D. Piensa en el audio volando sobre de ti, trepando atras de ti, moviéndose enfrente de ti" \url{https://developer.mozilla.org/en-US/docs/Web/API/Web_Audio_API/Web_audio_spatialization_basics} (Consultado el \today)}, puede arrojar algunas ideas sobre la actividad realizada en el marco de \textit{Panorama}. % Falta desarrollar esto 

\color{black}

El diseño del módulo de \textit{streaming} presentó una discusión sobre lo legal, la reproducción de audio / video en la web y el uso de recursos. Las restricciones que imponen servidores privados como YouTube o Facebook Live al contenido remixeado fue una de las motivaciones para implementar un sistema personalizado. Con esto, el sistema evita el mercado exclusivo de servicios de streaming privados, con licencias de paga o limitados. De manera complementaria, el sistema de \textit{streaming} personalizado permite controlar la cantidad de recursos usados y usuarios conectados. Los formatos para la transmisión de video en web explorados fueron: RTMP (Real Time Messaging Protocol), FLV (Flash Video), HLS (HTTP Live Streaming), video a través de WebSockets\footnote{``WebSockets es una tecnología avanzada que hace posible abrir una sesión de comunicación interactiva entre el navegador del usuario y un servidor.'' \url{https://developer.mozilla.org/es/docs/Web/API/WebSockets_API}(Consultado el \today)}, WebRTC (Web Real-Time Communication) y MPEG-DASH (Dynamic Adaptive Streaming over HTTP).

Las características actuales presentes en HTML5 (Quinta versión de HyperText Markup Language) permiten utilizar formatos de streaming no nativos con \textit{frameworks} de decodificación para el despliegue de video en formatos soportados por el navegador. \textit{Panorama} utiliza RTMP debido a 1) la robustez del protocolo, 2) la rapidez de transmisión de datos, 3) la documentación que existe sobre NGINX\footnote{``Nginx es un servidor web/proxy inverso ligero de alto rendimiento". \url{https://es.wikipedia.org/wiki/Nginx}(Consultado el \today)} RTMP y 4) el uso de FFmpeg\footnote{``FFmpeg es el marco de trabajo multimedia líder capaz de decodificar, codificar, transcodificar, mux, demux, transmitir, filtrar y reproducir casi cualquier cosa que los humanos o las máquinas hayan creado". \url{https://www.ffmpeg.org/about.html}(Consultado el \today)} para la recodificación y redimensión de video en la web. La plataforma utiliza FLV debido a la velocidad del protocolo (no se descarta el uso de otros protocolos en el futuro). Para el uso de FLV en el navegador, se utilizó FLV.js. 

Para la interacción entre usuarios se utilizó WebSockets. Esto implicó: 1) uso de chat, 2) compartición en tiempo real de eventos gestuales (rotación y posición) del avatar utilizado y 3) personalización de modelo, textura y nombre de cada usuario. 

Del lado del servidor se usaron balanceadores de carga junto con modelos de configuración para la instalación y eliminación de recursos, haciendo independientes los servidores dedicados al \textit{streaming} y a servicios web. Esto permite que los espacios puedan ser escritos, eliminados o conservados a voluntad. 

% Escribir sobre las peculiaridades del git que estamos referenciando 

Los espacios escritos consistieron en incipientes modos de exploración copresencial que guardaban una relación entre la posición y la rotación de la cámara, controlada por medio del ratón y el teclado de la computadora. Para el control se utlizaron las teclas WASD y flechas de navegación siguiendo la convención de la cámara de videojuegos en primera persona. La información de posición y rotación fueron compartidas y podían visualizar el movimiento de avatares que representaban a los asistentes en el espacio digital. 

% Cuerpo digital 

La concatenación de software estuvo delimitado por la experiencia de usuarios hipotéticos. Esto nos llevo a plantear los posibles agentes que se involucraban con la experiencia inmersiva. Como la propuesta partió de eventos performáticos tipo concierto, la figura de intérpretes/artistas fue central. El mantenimiento del flujo performático expresado a partir de una transmisión de audio y video delimitó el aspecto técnico y estético del espacio: El primer objetivo a resolver fue el streaming que pudiera ser estable, eficiente, compatible con la convención de transmisión personalizada ejecutada con programas de interfaz gráfica como OBS (Open Broadcaster Software) o butt (broadcast using this tool).

El segundo tipo de agente fue el público que navegaba en el espacio y que realizaba gestos corporales virtuales sencillos: posición y rotación. 

Un tercer tipo de agente se involucró con los aspectos fuera del escenario, la lógistica de la transmisión y el mantenimiento del sistema en tiempo real visibilizó la figura del \textit{staff}. En este sentido la labor tecnológica virtual del equipo de \textit{PiranhaLab} fue cercana a la logística y a la división del trabajo de un escenario para conciertos presenciales. 
