
\section*{Contemplación y EDGES} % Espacio para hablar del proyecto de (acomplete) de Dorian

\begin{itemize}
\item Distopía
\item domo 
\item Underborders
\item Milena y Concepción
\item setInterval()
\item Contemplación del fin del Mundo
\item sistemas mixtos
\item espacio y performance fusionados en Contemplación
\end{itemize}


La contemplación del fin del mundo es un performance a modo de ejuego por ser el último evento de la serie de conciertos edges, dando por finalizado de manera simbólica, destruyendo el escenario.
Los asistentes podían presenciar el fin del mundo a modo de destrucción del escenario, inundaciones, objetos celestiales y finalmente la dispersión de los colores del escenario, dejando los objetos sin razgos reconocibles.

La idea principal sirvió como vehículo para la exploración del espacio como característica del performance, el mundo explorable, la persecución en forma de figuras celestiales que ocupaban todo el espacio o que se expandían e iluminaban todo así como las inundaciociones, además del uso de pantallas distribuidas a lo largo de todo el mundo, permitiendo a los usuarios presenciar el performance desde cualquier ubicación.

Se puede destacar de este concierto, el uso de acciones colectivas lanzadas por el artista, quien durante el trascurso del evento podía cambiar las características del ambiente de manera similar entre los participantes, teniendo todos una experiencia compartida homogénea y en el momento, además del uso de framework externos  para la creación de visuales.
Las dificultades de las experiencias compartidas radican en la sincronización de eventos, tanto para los usuarios que ingresan desde el inicio o los usuarios ocasionales, sin importar ubicación geográfica o dispositivo.
Esta posibilidad agrega la capacidad de interacción de parte del artista, generando situaciones que añadan dinámica al juego, donde los asistentes podrían participar más allá de ser meros observadores
