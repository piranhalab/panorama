
\section*{Discusión}

Propuesta para esta sección: Desmantelar e integrar en resultados, descripción y conclusiones

\begin{itemize}

  \color{black}

\item {Black - Dejar en conclusiones}

  \color{BlueViolet}

\item BlueViolet - Para Contemplación 
  
  \color{Fuchsia}
  
\item Fuschia - Para Notas de Ausencia

  \color{MidnightBlue}

\item MidnightBlue - Para Anti

  \color{BlueGreen}

\item BlueGreen - Para Descripción tecnológica
  
\end{itemize}

\color{Fuchsia}

La investigación detectó funcionalidad y experimentación como dos posibilidades de un continuo para la escritura de software en un marco artístico y performático. EDGES como plataforma explicita el papel experimental de los actos, la plataforma tecnológica también podría ser experimental e incluso podría desdibujarse en pos de la integración performance-espacio bajo la misma premisa de la experimentación.

\color{black}

Los eventos realizados en esta diversidad de plataformas han utilizado ligas a internet que de acuerdo a la fecha consultada, redireccionan a distintos espacios virtuales. A diferencia de los sitios que utilizan texto y entornos de programación web como HTML, la mezcla de módulos y el uso de frameworks dedicados que utilizan renderizadores 3d como webGL, motores de audio como Web Audio API o plataformas de transmisión de audio y video personalizadas y efímeras dificultan la documentación convencional. La labor se complica cuando el mantenimiento de estos espacios sobrepasa los alcances temporales o económicos del proyecto. El reto metodológico que esto supone es un asunto pendiente para las investigaciones que hacen referencia a tecnología. En este sentido, la referencia a repositorios de código públicos podrían arrojar soluciones para la documentación y arqueología de los desarrollos tecnológicos. Una alternativa para la documentación de estos procesos es \textit{Wayback Machine}\footnote{``\textit{The Wayback Machine} es una iniciative de Internet Archive para construir una librería digital de sitios de Internet y otros artefactos culturales en formato digital''. \url{http://web.archive.org/} (Consultado el \today)}. 

\color{BlueViolet}

Una de las reflexiones que estuvo presente desde el inicio del proyecto partió de las distinciones más inmediatas: mundos virtuales / mundos reales, materialidad / inmaterialidad. Consideramos que procesos como los de \textit{Panorama} transitan entre la extensión de la fisicalidad hacia la virtualidad\footnote{`la trascendencia de la fisicalidad del mundo virtual nos permite extender nuestro modo de operación en el mundo físico. Nuevas formas de viaje, una nueva forma de comunicación, una nueva forma de operación, un nuevo medio de expresión" pp. 49} \citep{cyberspace} y la reconfiguración de la materialidad\footnote{``el arte no solamente podría ser performeado en el plano sensorial, sino también en el plano inteligible. Las estéticas de la participación y la constitución de los medios digitales podrían intepretarse como la continuación de algunos de esos prinicipios." pp 190} \citep{andreasosa} de cara al giro digital que posibilita la continuación y contraposición estéticas y formas de organización social. 

\color{MidnightBlue}

Señalamos la naturaleza efímera de obras y espacios en la lógica digital, y consideramos que el reto técnico, investigativo y de (re)activación está abierto y en constante discusión. En este sentido, la formación de artistas capaces de interactuar y ejecutar estas piezas podría apuntar a la consideración de estos aspectos en los programas de interpretación musical con nuevas tecnologías. Adicionalmente el aspecto formativo podría extenderse hacia la investigación y escritura de software.  % Indagar sobre esto

% Aquí puede ir lo de los espacios y arquitecturas. También aquí puede ir lo de materialidad y neomaterialidad 

\color{BlueGreen}

La presente investigación estuvo relacionada con la gamificación emergente\footnote{O Ludificación, del inglés \textit{gamification}. Para \cite{gamificacion} existe la gamificación intencional y emergente. El presente artículo hace referencia a la segunda: ``la gamificación se puede definir como un proceso cultural gradual y emergente, aunque no intencional, derivada del compromiso cada vez más generalizado con los juegos e interacciones lúdicas."} tácita en la escritura de proyectos como \textit{Panorama}. Esto implica 1) tecnología, por ejemplo pantallas, combinaciones de teclas para la exploración de espacios pero también dispositivos de realidad virtual, 2) diseño visual y sonoro de los espacios, objetos digitales en espacios tridimensionales, audio inmersivo, 3) intercambio de información para la co-presencia en tiempo real dentro de espacios digitales o el multijugador e incluso 4) narrativas e imaginarios convenidos, explorados desde el diseño de la experiencia y esperados del lado del público. 

% Referencias a la consecuencia y no al objeto por sí mismo 

% La discusión en torno a la escritura de software inicia con la perspectiva prostética y termina problematizando las ideas que orbitan en torno a la cajanegrización de cara a la idea del trabajo socialmente invertido en esta actividad.

\color{BlueViolet}

%El término cyberespacio\footnote{``El término cyberespacio puede ser usado vagamente para describir la interconexión espacial e invisible de computadoras en el Internet y es aplicado a casi cualquier experiencia espacial y virtual creada en una computadora'' \citep{cyberspace}} puede ser un punto de partida para describir la actividad relacionada con \textit{Panorama}. Puede ser contrapuesto con diversos conceptos que coinciden en la práctica, en la arquitectura de intercambio de información, en la denominación conceptual y en la convergencia de tecnologías expresamente mercantilizadas para la distribución de experiencias de Realidad Virtual (VR) como Oculus Rift. VR también puede ser un concepto inicial, socialmente convenido para describir la experiencia de \textit{Panorama} pero para evitar contradicciones en la conducción de una experiencia en el cyberespacio hacia exploraciones hacia dispositivos/interfaces para la interacción en estos espacios, el presente artículo decide retomar la noción de inmersividad. Consideramos que la relación entre cyberespacio como una noción narrativa y de inmersividad como una forma de resolución técnica de exploración de una cámara en un espacio tridimensional, con audio posicionado\footnote{``En \textit{Web Audio}, espacializaciones complejas en 3D son creadas utilizando PannerNode, que en términos legos básicamente son muchas matemáticas \textit{cool} para hacer que el audio aparezca en un espacio 3D. Piensa en el audio volando sobre de ti, trepando atras de ti, moviéndose enfrente de ti" \url{https://developer.mozilla.org/en-US/docs/Web/API/Web_Audio_API/Web_audio_spatialization_basics} (Consultado el \today)}, puede arrojar algunas ideas sobre la actividad realizada en el marco de \textit{Panorama}. % Falta desarrollar esto 

\color{Fuchsia}

Manifiestos, posturas políticas y alternativas en la organización que dialogan con la escritura de software como desarrollo tecnológico y como acto creativo. Por ejemplo \textit{live coding} y la transparencia de los procesos o el uso de interfaces de texto \citep{collinsLivecoding}, el manifiesto de una servidora feminista \citep{feministserver} o la arquitectura de distribución de información par a par\footnote{\textit{P2P} (par a par) por sus siglas en inglés.  ``La arquitectura de una red distribuida puede ser llamada Par a Par (P-to-P, P2P, ...)   si los participantes comparten una parte de los recursos de su propio software (poder de procesamiento, capacidad de almacenamiento, capacidad de conexión a la red, impresoras,...) Estos recursos compartidos son necesarios para proveer el Servicio y el contenido ofrecido por la red... Estos son accedidos por otros pares directamente sin pasar por entidades intermediarias." \citep{p2p}} que persigue la distribución y la descentralización en redes que posibilitan espacios virtuales \citep{cyberspace} y que incluso puede extenderse al autocuidado y formas alternativas de expresar relaciones sociales en red \citep{dwc}. 

\color{black}

Estas perspectivas pueden extenderse hacia una postura para la investigación de tecnología y el papel que juegan en la política de los espacios físicos y virtuales, como el cuarto propio \citep{cuartopropio} o el buen conocer (cita platohedro). Podríamos relacionar estos procesos con el giro de los nuevos medios descrito por \cite{manovichlanguage}, las implicaciones sociales de este giro y sobre todo, las consecuencias estéticas que a partir de este se abren y desenvuelven en el performance musical por medio de la computadora y otras prácticas afines. 

\color{MidnightBlue}

De este mapa ¿Podríamos establecer un corpus de conceptos y estéticas \citep{shankenCanon} que pudieran explicar las nuevas prácticas artísticas con medios que implican a la música pero también a otros campos, que además pudiera considerar aspectos sociales y políticos implicados en la música y expresiones cercanas a ésta? ¿Cómo se vincularía este giro con las implicaciones políticas y estéticas de la programación de software y su ejecución \citep{speakingCode} de cara a expresiones más claramente delimitadas hacia la música y la tecnología que la posibilita?

\iffalse

- manifiestos y posturas políticas > live coding y servidora feminista 
- Diferencias con respecto a otros espacios
- Discusión sobre lo digital, los nuevos medios y la virtualidad. 
- Giro de los nuevos medios
- Agotamiento del formato
- funcionalidad - experimentación 
- Arqueología en internet (cosas que ya tienen tiempo y cosas recientes). 
- El papel de los videojuegos gamización
- Lo barroco y el artículo de electroacústica. 
- Espacio, arquitectura e integración: puede el espacio virtual convertirse en un objeto con agencia en la interpretacion en vivo? 

\fi

\color{black}
