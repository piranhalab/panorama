
\section*{Discusión}

Propuesta para esta sección: Desmantelar e integrar en resultados, descripción y conclusiones

\begin{itemize}

  \color{black}

\item {Black - Dejar en conclusiones}

  \color{BlueViolet}

\item BlueViolet - Para Contemplación 
  
  \color{Fuchsia}
  
\item Fuschia - Para Notas de Ausencia

  \color{MidnightBlue}

\item MidnightBlue - Para Anti

  \color{BlueGreen}

\item BlueGreen - Para Descripción tecnológica
  
\end{itemize}

% Aquí puede ir lo de los espacios y arquitecturas. También aquí puede ir lo de materialidad y neomaterialidad 

% Referencias a la consecuencia y no al objeto por sí mismo 

% La discusión en torno a la escritura de software inicia con la perspectiva prostética y termina problematizando las ideas que orbitan en torno a la cajanegrización de cara a la idea del trabajo socialmente invertido en esta actividad.

\color{BlueViolet}

%El término cyberespacio\footnote{``El término cyberespacio puede ser usado vagamente para describir la interconexión espacial e invisible de computadoras en el Internet y es aplicado a casi cualquier experiencia espacial y virtual creada en una computadora'' \citep{cyberspace}} puede ser un punto de partida para describir la actividad relacionada con \textit{Panorama}. Puede ser contrapuesto con diversos conceptos que coinciden en la práctica, en la arquitectura de intercambio de información, en la denominación conceptual y en la convergencia de tecnologías expresamente mercantilizadas para la distribución de experiencias de Realidad Virtual (VR) como Oculus Rift. VR también puede ser un concepto inicial, socialmente convenido para describir la experiencia de \textit{Panorama} pero para evitar contradicciones en la conducción de una experiencia en el cyberespacio hacia exploraciones hacia dispositivos/interfaces para la interacción en estos espacios, el presente artículo decide retomar la noción de inmersividad. Consideramos que la relación entre cyberespacio como una noción narrativa y de inmersividad como una forma de resolución técnica de exploración de una cámara en un espacio tridimensional, con audio posicionado\footnote{``En \textit{Web Audio}, espacializaciones complejas en 3D son creadas utilizando PannerNode, que en términos legos básicamente son muchas matemáticas \textit{cool} para hacer que el audio aparezca en un espacio 3D. Piensa en el audio volando sobre de ti, trepando atras de ti, moviéndose enfrente de ti" \url{https://developer.mozilla.org/en-US/docs/Web/API/Web_Audio_API/Web_audio_spatialization_basics} (Consultado el \today)}, puede arrojar algunas ideas sobre la actividad realizada en el marco de \textit{Panorama}. % Falta desarrollar esto 



\iffalse

- manifiestos y posturas políticas > live coding y servidora feminista 
- Diferencias con respecto a otros espacios
- Discusión sobre lo digital, los nuevos medios y la virtualidad. 
- Giro de los nuevos medios
- Agotamiento del formato
- funcionalidad - experimentación 
- Arqueología en internet (cosas que ya tienen tiempo y cosas recientes). 
- El papel de los videojuegos gamización
- Lo barroco y el artículo de electroacústica. 
- Espacio, arquitectura e integración: puede el espacio virtual convertirse en un objeto con agencia en la interpretacion en vivo? 

\fi

\color{black}
