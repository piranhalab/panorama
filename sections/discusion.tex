
\section*{Discusión}

Los eventos realizados en esta diversidad de plataformas han utilizado ligas a internet que de acuerdo a la fecha consultada, redireccionan a distintos espacios virtuales. A diferencia de los sitios que utilizan texto y entornos de programación web como HTML, la mezcla de módulos y el uso de frameworks dedicados que utilizan renerizadores 3d como webGL, motores de audio como Web Audio API o plataformas de transmisión de audio y video personalizadas y efímeras dificultan la documentación convencional. La labor se complica cuando el mantenimiento de estos espacios sobrepasa los ealcances temporales o económicos del proyecto. El reto metodológico que esto supone es un asunto pendiente para las investigaciones que hacen referencia a tecnología. En este sentido, la referencia a repositorios de código públicos podrían arrojar soluciones para la documentación y arqueología de los desarrollos tecnológicos. Una alternativa para la documentación de estos procesos es Wayback Machine.  % Indagar sobre esto

La discusión en torno a la escritura de software inicia con la perspectiva prostética y termina problematizando las ideas que orbitan en torno a la cajanegrización de cara a la idea del trabajo socialmente invertido en esta actividad.
 
\iffalse

- Diferencias con respecto a otros espacios
- Discusión sobre lo digital, los nuevos medios y la virtualidad. 
- Giro de los nuevos medios
- Agotamiento del formato
- ¿Se cumplió la premisa? Pretexto para abrir cajas negras y plantear discusiones
- funcionalidad - experimentación 
- Arqueología en internet (cosas que ya tienen tiempo y cosas recientes). 
- El papel de los videojuegos gamización
- Lo barroco y el artículo de electroacústica. 

\fi
