
\section*{Discusión}

Los eventos realizados en esta diversidad de plataformas han utilizado ligas a internet que de acuerdo a la fecha consultada, redireccionan a distintos espacios virtuales. A diferencia de los sitios que utilizan texto y entornos de programación web como HTML, la mezcla de módulos y el uso de frameworks dedicados que utilizan renderizadores 3d como webGL, motores de audio como Web Audio API o plataformas de transmisión de audio y video personalizadas y efímeras dificultan la documentación convencional. La labor se complica cuando el mantenimiento de estos espacios sobrepasa los alcances temporales o económicos del proyecto. El reto metodológico que esto supone es un asunto pendiente para las investigaciones que hacen referencia a tecnología. En este sentido, la referencia a repositorios de código públicos podrían arrojar soluciones para la documentación y arqueología de los desarrollos tecnológicos. Una alternativa para la documentación de estos procesos es Wayback Machine.  % Indagar sobre esto

La presente investigación tuvo presente la gamificación emergente\footnote{O Ludificación, del inglés \textit{gamification}. Para \cite{gamificacion} existe la gamificación intencional y emergente. El presente artículo hace referencia a la segunda: ``la gamificación se puede definir como un proceso cultural gradual y emergente, aunque no intencional, derivada del compromiso cada vez más generalizado con los juegos e interacciones lúdicas."} tácita en la escritura de proyectos como \textit{Panorama}. Esto implica 1) tecnología, por ejemplo pantallas, combinaciones de teclas para la exploración de espacios pero también dispositivos de realidad virtual, 2) diseño visual y sonoro de los espacios, objetos digitales en espacios tridimensionales, audio inmersivo, 3) intercambio de información para la co-presencia en tiempo real dentro de espacios digitales o el multijugador e incluso 4) narrativas e imaginarios convenidos, explorados desde el diseño de la experiencia y esperados del lado del público. 

% Referencias a la consecuencia y no al objeto por sí mismo 

% La discusión en torno a la escritura de software inicia con la perspectiva prostética y termina problematizando las ideas que orbitan en torno a la cajanegrización de cara a la idea del trabajo socialmente invertido en esta actividad.

Manifiestos, posturas políticas y alternativas en la organización que dialogan con la escritura de software como desarrollo tecnológico y como acto creativo. Por ejemplo \textit{live coding} y la transparencia de los procesos o el uso de interfaces de texto \citep{collinsLivecoding}, el manifiesto de una servidora feminista \citep{feministserver} o la arquitectura de distribución de información par a par\footnote{\textit{P2P} (par a par) por sus siglas en inglés.  ``La arquitectura de una red distribuida puede ser llamada Par a Par (P-to-P, P2P, ...)   si los participantes comparten una parte de los recursos de su propio software (poder de procesamiento, capacidad de almacenamiento, capacidad de conexión a la red, impresoras,...) Estos recursos compartidos son necesarios para proveer el Servicio y el contenido ofrecido por la red... Estos son accedidos por otros pares directamente sin pasar por entidades intermediarias." \citep{p2p}} que persigue la distribución y la descentralización y que incluso puede extenderse al autocuidado y formas alternativas de expresar relaciones sociales en red \citep{dwc}. 

Estas perspectivas pueden incluso extenderse hacia una postura para la investigación de tecnología y el papel que juegan en la política de los espacios físicos y virtuales, como el cuarto propio \citep{cuartopropio} o el buen conocer (cita platohedro).  

Funcionalidad y experimentación como dos posibilidades de un continuo para el la escritura de software en un marco artístico y performático. EDGES como plataforma explicita el papel experimental de los actos, la plataforma tecnológica también podría ser experimental e incluso podría desdibujarse en pos de la integración performance-espacio bajo la misma premisa de la experimentación. 


 
\iffalse

- manifiestos y posturas políticas > live coding y servidora feminista 
- Diferencias con respecto a otros espacios
- Discusión sobre lo digital, los nuevos medios y la virtualidad. 
- Giro de los nuevos medios
- Agotamiento del formato
- funcionalidad - experimentación 
- Arqueología en internet (cosas que ya tienen tiempo y cosas recientes). 
- El papel de los videojuegos gamización
- Lo barroco y el artículo de electroacústica. 
- Espacio, arquitectura e integración: puede el espacio virtual convertirse en un objeto con agencia en la interpretacion en vivo? 

\fi
