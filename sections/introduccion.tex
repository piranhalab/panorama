
\section*{Ecosistema}

% Repositorio de panorama en construcción 

\textit{Panorama} \citep{panorama} fue un programa escrito por \textit{PiranhaLab}\footnote{``PiranhaLab es un laboratorio interdisciplinario que trabaja en las tripas del software''. \url{https://piranhalab.github.io/} (Consultado el \today)}, implementado en el marco de eventos institucionales, independientes y comisiones específicas. La hipótesis/premisa central de este proyecto buscó que los usuarios pudieran compartir una experiencia ligera para el navegador de manera co-presencial, aprovechando las posibilidades de las tecnologías de transmisión de audio y video. % Primero la hipotesis de trabajo  
Hemos decidido delimitar el desarrollo de Panorama en torno a estos eventos y en particular, a \textit{EDGES}\footnote{``Plataforma de experimentación y difusión de proyectos audiovisuales en vivo'' impulsada por el Laboratorio de Imágenes en Movimiento del Centro Multimedia del Centro Nacional de las Artes. \url{https://www.facebook.com/events/209679013466792} (Consultado el \today)}. Los eventos realizados en el marco de este ciclo se llevaron a cabo del 6 de agosto al 19 de noviembre de 2020. Algunos eventos independientes sucedieron ligeramente antes o después de estas fechas. 	

Este proyecto de investigación considera una selección de eventos como casos de estudio para la comprobación de la hipótesis/premisa principal del software desarrollado: \textit{Notas de Ausencia}\footnote{\url{https://notasdeausencia.cc} (Consultado el \today)} de Marianne Teixido, \textit{La Contemplación del Fin del Mundo}\footnote{\url{https://edges.piranhalab.cc} (Consultado el \today)} de Dorian Sotomayor y \textit{threecln}\footnote{\url{https://threecln.piranhalab.cc}(Consultado el \today)} de Emilio Ocelotl. Complementamos la descripción con algunos espacios y eventos adicionalmente seleccionados: \textit{Pruebas Proféticas}, \textit{Distopia}, \textit{Underborders} y \textit{4NT1}. % Esto es provisional, por lo menos son los que referencio más abajo

El artículo aporta elementos a una discusión para la reflexión transversal (técnica, estética y  de investigación académica) y utiliza conceptos que nos permiten desplazarnos entre estos hilos para entretejer la investigación transdisciplinaria. se adscribe a los planteamientos de los estudios del software y busca extender la discusión del terreno técnico y descriptivo. A lo largo del texto buscamos problematizar el papel que juega la computadora (local o en servidores) en la realización de actividades musicales y artísticas.

%% La discusión inicia con la idea del software como una prótesis y termina problematizando las ideas que orbitan en torno a la cajanegrización de cara a la idea del trabajo socialmente invertido para la escritura de software. % Esto puede ir en discusión 

% Pendiente: cómo resolvemos el problema de ser gestores, artistas y escritores. Investigación artística

% Segundo pendiente: realmente tenemos la capacidad narrativa de plantear una navegación 
