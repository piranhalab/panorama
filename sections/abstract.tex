
El confinamiento provocado por la pandemia de COVID-19 obligó a artistas, gestores, instituciones públicas e industrias a replantear maneras de compartir flujos co-presenciales y hacer performance audiovisual en vivo.

El presente artículo describe \textit{Panorama}, un conjunto de módulos de código y software que permiten realizar conciertos en espacios virtuales y tridimensionales alojados en la web. De manera complementaria, enuncia discusiones que surgieron durante la activación del espacio sobre materialidad, virtualidad, descentralización, distribución, espacio público, reconocimiento y ejecución telemática.                            

