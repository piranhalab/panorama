
El confinamiento provocado por la pandemia de COVID-19 obligó a artistas, gestores, instituciones públicas e industrias a replantear maneras de compartir flujos co-presenciales y hacer performance audiovisual en vivo. La presente investigación se enmarca en está búsqueda; el distanciamiento social forzado fue el pretexto para resolver necesidades tecnológicas que pudieran equilibrar la perspectiva funcional y la experimental. Este trayecto desembocó en discusiones sobre materialidad y virtualidad.                          

El presente artículo describe \textit{Panorama}\footnote{Repositorio de Panorama.}, un conjunto de módulos de código y software que permiten realizar conciertos en espacios virtuales, tridimensionales alojados en la web. La hipótesis/premisa central de este proyecto buscó que los usuarios pudieran compartir una experiencia ligera para el navegador de manera co-presencial, aprovechando las posibilidades de las tecnologías de transmisión de audio y video.  
