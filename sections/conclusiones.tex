
\section*{Conclusiones}

% La pregunta principial sería: ¿Se cumplió la hipótesis/premisa del software escrito? 

Los usuarios pudieron compartir una experiencia ligera para el navegador de manera co-presencial, aprovechando las posibilidades de las tecnologías de transmisión de audio y video. Destacamos la importancia de plantear soluciones compartibles en lo que respecta a la transmisión de audio-imagen. La escritura de esta parte de \textit{Panorama} fue utilizado en el marco de otros eventos y ciclos.

%El proyecto no tuvo un plan de acción específico para la realización de mediciones durante los conciertos, a pesar de que el control del servidor lo permitió. Sin embargo, la lectura de información a partir de una simulación del evento arroja datos importantes en este sentido.

%Sobre las impresiones del público, fue posible realizar un cuestionario durante \textit{Pruebas Proféticas}. 

%La importancia de una infraestructura tecnológica y social. Esto se puede entretejer a partir de decisiones curatoriales que vinculan potencial tecnológico con alianzas entre actores. 

%Del lado teórico es necesario convenir y reforzar un corpus de conceptos para la investigación sobre Tecnología Musical. Detección de campos y persectivas de investigación: las agencias que están presentes en eventos que se desenvuelven con tecnología.


Los eventos realizados en la diversidad de plataformas anteriormente descrita han utilizado ligas a internet que de acuerdo a la fecha consultada, redireccionan a distintos espacios virtuales. A diferencia de los sitios que utilizan texto y entornos de programación web como HTML, la mezcla de módulos y el uso de frameworks dedicados que utilizan renderizadores 3d como webGL, motores de audio como Web Audio API o plataformas de transmisión de audio y video personalizadas y efímeras dificultan la documentación convencional. La labor se complica cuando el mantenimiento de estos espacios sobrepasa los alcances temporales o económicos del proyecto. El reto metodológico que esto supone es un asunto pendiente para las investigaciones que hacen referencia a tecnología. En este sentido, la referencia a repositorios de código públicos podrían arrojar soluciones para la documentación y arqueología de los desarrollos tecnológicos. Una alternativa para la documentación de estos procesos es \textit{Wayback Machine}\footnote{``\textit{The Wayback Machine} es una iniciative de Internet Archive para construir una librería digital de sitios de Internet y otros artefactos culturales en formato digital''. \url{http://web.archive.org/} (Consultado el \today)}. 

\color{black}

Estas perspectivas pueden extenderse hacia una postura para la investigación de tecnología y el papel que juegan en la política de los espacios físicos y virtuales, como el cuarto propio \citep{cuartopropio} o el buen conocer (cita platohedro). Podríamos relacionar estos procesos con el giro de los nuevos medios descrito por \cite{manovichlanguage}, las implicaciones sociales de este giro y sobre todo, las consecuencias estéticas que a partir de este se abren y desenvuelven en el performance musical por medio de la computadora y otras prácticas afines. 

Destacamos el giro de los nuevos medios como un cambio de paradigma que la tecnología musical debe considerar por las implicaciones de los usos/críticas de las tecnologías de la información, las tendencias para la resolución de problemas relacionados con gestión de datos y las consecuencias estéticas que a veces se empalman y otras rebasan a la música y las perspectivas de investigación asociadas a esta disciplina. El rodeo o la realización de un motivo tecnológico como una perspectiva de investigación que pueda aportar en el aspecto tecnológico y teórico-metodológico, sobre todo en campos que lo permiten como humanidades, artes y específicamente, investigación que implica musica y expresiones audiovisuales con tecnología.
