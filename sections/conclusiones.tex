
\section*{Conclusiones}

% La pregunta principial sería: ¿Se cumplió la hipótesis/premisa del software escrito? 

Los usuarios pudieron compartir una experiencia ligera para el navegador de manera co-presencial, aprovechando las posibilidades de las tecnologías de transmisión de audio y video. Destacamos la importancia de plantear soluciones compartibles en lo que respecta a la transmisión de audio-imagen. La escritura de esta parte de \textit{Panorama} fue utilizado en el marco de otros eventos y ciclos.

El proyecto no tuvo un plan de acción específico para la realización de mediciones durante los conciertos, a pesar de que el control del servidor lo permitió. Sin embargo, la lectura de información a partir de una simulación del evento arroja datos importantes en este sentido.

Sobre las impresiones del público, fue posible realizar un cuestionario durante \textit{Pruebas Proféticas}. 

La importancia de una infraestructura tecnológica y social. Esto se puede entretejer a partir de decisiones curatoriales que vinculan potencial tecnológico con alianzas entre actores. 

Del lado teórico es necesario convenir y reforzar un corpus de conceptos para la investigación sobre Tecnología Musical. Detección de campos y persectivas de investigación: las agencias que están presentes en eventos que se desenvuelven con tecnología.

Destacamos el giro de los nuevos medios como un cambio de paradigma que la tecnología musical debe considerar por las implicaciones de los usos/críticas de las tecnologías de la información, las tendencias para la resolución de problemas relacionados con gestión de datos y las consecuencias estéticas que a veces se empalman y otras rebasan a la música y las perspectivas de investigación asociadas a ésta disciplina. 

El rodeo o la realización de un motivo tecnológico como una perspectiva/metodología de investigación que pueda aportar en el aspecto tecnológico y teórico-metodológico, sobre todo en campos que lo permiten como humanidades, artes y específicamente, investigación artística con tecnología
