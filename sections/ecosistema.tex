
\section*{Ecosistema}

% Repositorio de panorama en construcción 

\textit{Panorama} \citep{panorama} fue un programa escrito por \textit{PiranhaLab}\footnote{``PiranhaLab es un laboratorio interdisciplinario que trabaja en las tripas del software''. \url{https://piranhalab.github.io/} (Consultado el \today)}, implementado en el marco de eventos institucionales, independientes y comisiones específicas. La hipótesis/premisa central de este proyecto buscó que los usuarios pudieran compartir una experiencia ligera para el navegador de manera co-presencial, aprovechando las posibilidades de las tecnologías de transmisión de audio y video. % Primero la hipotesis de trabajo  
Hemos decidido delimitar el desarrollo de Panorama en torno a estos eventos y en particular, a \textit{EDGES}\footnote{``Plataforma de experimentación y difusión de proyectos audiovisuales en vivo'' impulsada por el Laboratorio de Imágenes en Movimiento del Centro Multimedia del Centro Nacional de las Artes. \url{https://www.facebook.com/events/209679013466792} (Consultado el \today)}. Los eventos realizados en el marco de este ciclo se llevaron a cabo del 6 de agosto al 19 de noviembre de 2020. Algunos eventos independientes sucedieron ligeramente antes o después de estas fechas. 	

Este proyecto de investigación considera una selección de eventos como casos de estudio para la comprobación de la hipótesis/premisa principal del software desarrollado: \textit{Notas de Ausencia}\footnote{\url{https://notasdeausencia.cc} (Consultado el \today)} de Marianne Teixido, \textit{La Contemplación del Fin del Mundo}\footnote{\url{https://edges.piranhalab.cc} (Consultado el \today)} de Dorian Sotomayor y \textit{threecln}\footnote{\url{https://threecln.piranhalab.cc}(Consultado el \today)} de Emilio Ocelotl. Complementamos la descripción con algunos espacios y eventos adicionalmente seleccionados: \textit{Pruebas Proféticas}, \textit{Distopia}, \textit{Underborders} y \textit{4NT1}. % Esto es provisional, por lo menos son los que referencio más abajo

El artículo aporta elementos a una discusión para la reflexión transversal (técnica, estética y  de investigación académica) y utiliza conceptos que nos permiten desplazarnos entre estos hilos para entretejer la investigación transdisciplinaria. Se adscribe a los planteamientos de los estudios del software y busca extender la discusión del terreno técnico y descriptivo. A lo largo del texto buscamos problematizar el papel que juega la computadora (local o en servidores) en la realización de actividades musicales y peformáticas.

%% La discusión inicia con la idea del software como una prótesis y termina problematizando las ideas que orbitan en torno a la cajanegrización de cara a la idea del trabajo socialmente invertido para la escritura de software. % Esto puede ir en discusión 

% Pendiente: cómo resolvemos el problema de ser gestores, artistas y escritores. Investigación artística

% Segundo pendiente: realmente tenemos la capacidad narrativa de plantear una navegación 

% \subsection*{Ecosistema} % Esto antes se llamaba antecedentes ivaciones paralelasun titulo más serio sería: activaciones paralelas

La primera parte de este apartado describe una serie de plataformas, paralelas a\textit{Panorama}, que permitieron realizar conciertos audiovisuales en el navegador con entornos tridimensionales. Los puntos de coincidencia entre espacios que la investigación detectó fueron: audio y video transmitido en tiempo real y la posibilidad de posicionar pantallas, audio, avatares y escenarios en el espacio. La alusión a las plataformas que compartieron ecosistema con \textit{Panorama} se delimita proyectos con cercanía performática, la mayoría de ellos escritos e implementados en la Ciudad de México. 

En el contexto de la programación al vuelo o \textit{live coding}, los Algoraves organizados por Algo:ritmi\footnote{\url{https://www.facebook.com/AlgoritmiTorino/about/} (Consultado el \today)} iniciaron el interés por los espacios tridimensionales de realidad virtual para lidiar con el distanciamiento social de la pandemia. Estos eventos tuvieron lugar en Mozilla Hubs\footnote{\url{https://hubs.mozilla.com/} (Consultado el \today)}. Esta plataforma resuelve el backend de la experiencia y permite al diseñador de espacio centrarse en el frontend, el montaje del escenario al que acceden los usuarios.

% El diseño de espacios de arquitecturas colosales fue un rasgo distintivo de estos eventos. % esto suena medio chistoso 

Para el caso de la comunidad creativa en México, algunos otros casos de implementación de espacios virtuales en situaciones de concierto fueron propuestas por TOPLAP México. De manera similar a Algo:ritmi, Algoraves eventos relacionados con la escena de la programación al vuelo\footnote{\url{https://networkmusicfestival.org/programme/performances/toplap-mexico-vr-algorave/} (Consultado el \today)} fueron organizados en FabricaVR, la plataforma de realidad virtual dedicada de TOPLAP México. Ambos casos forman parte de comunidades que antes de la pandemia, realizaban conciertos con tecnologías de transmisión de audio y video\footnote{\url{https://www.youtube.com/c/Eulerroom/videos}(Consultado el \today)}.% Nombres de las personas 

Adicionalmente, otras plataformas plantearon eventos similares con módulos de software distintos. Tal fue el caso de la plataforma OXXXO de Carlos Pesina que en co-participación con el festival Ceremonia permitió la realización de eventos digitales para afrontar la cancelación de conciertos presenciales.\footnote{La documentación es escasa. Ver: \url{https://www.instagram.com/p/B_bdw_TlrKa} (Consultado el \today) y \url{https://medium.com/@desyfree/musicaenvivocovid19-79e570a8f321} (Consultado el \today)}

\textit{A-frame}\footnote{\textit{A-frame} es ``un marco de trabajo para construir experiencias de realidad virtual(VR)''. \url{https://aframe.io/docs/1.2.0/introduction/} (Consultado el \today)} fue una solución alternativa a Mozilla Hubs para la realización de eventos virtuales. \textit{Sinestesia}\footnote{``Laboratorio de Experimentación, Improvisación y Nuevos Medios''. \url{https://www.instagram.com/si.nestes.ia/}(Consultado el \today)} y \textit{ZeYX Lab}\footnote{\url{https://zeyxlab.com/}(Consultado el \today)} implementaron este entorno. Three.js\footnote{``El proyecto de three.js apunta a la creación de una librería 3D fácil de usar, ligera, multinavegador, multipropósito''. \url{https://threejs.org/} (Consultado el \today)} fue otro \textit{framework} elegido para realizar este tipo de experiencias para el navegador. En este sentido destacamos el caso de \textit{Calindros}\footnote{\url{https://calindros.site/} (Consultado el \today)} de Hugo Solís. Este marco de trabajo resuelve el render gráfico, el resto de los complementos (comunicación entre pares, transmisión y procesamiento de señales de audio y video) debe ser resuelto de manera independiente. % Links caídos. Extender. Preguntar sobre el papel del juego. Falta un sitio más. Preguntar  

Finalmente, \textit{federacion-de-codigo-al-vuelo} \citep{en-vivo} fue el parteaguas de algunas ideas centrales que desembocaron en \textit{Panorama} pero también en otros dos proyectos: Zona hipermedial y Camposónico\footnote{\url{https://echoic.space/} (Consultado el \today)} \citep{camposonico}. La experiencia de colaboración con la \textit{federación} permitió concebir el entramado de módulos de código que pudieran resolver aspectos específicos. Tal es el caso de un chat o alguna forma de escritura en tiempo real o la transmisión de gestos de un avatar que extiendieran las posibilidades de desplazamiento y movimiento corporal en el espacio. La búsqueda por un espacio personalizado de los proyectos que se desprendieron de la \textit{federación} se estableció con el objetivo de resolver problemáticas de rendimiento que las plataformas de alto nivel resolvían de una manera poco transparente. 

Los eventos anteriormente enunciados resolvieron la transmisión de audio y video con plataformas privativas de streaming como YouTube y Twitch; plataformas para envío bidireccional de señales de audio como Sagora, jacktrip y Sonobus y plataformas para el montaje de servidores de audio y video como Icecast o LiquidSoap. Esta última experiencia apuntó a la escritura de un servidor personalizado de streaming de audio y video. % falta hablar de másservicios privativos de streaming

De la experiencia como asistentes, como ejecutantes y como incipientes investigadores/escritores de código en plataformas y eventos que implicaron estas resoluciones fue que el proyecto de \textit{PiranhaLab} empezó a delimitarse. Adicionalmente, sugieron discusiones sobre estos entornos, de las piezas y eventos que contuvieron e incluso del momento en el que espacio y e interpretación se desdibujan. 
