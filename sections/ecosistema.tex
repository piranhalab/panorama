
\section*{Ecosistema}

\textit{Panorama} \citep{panorama} fue un programa escrito por \textit{PiranhaLab}\footnote{``PiranhaLab es un laboratorio interdisciplinario que trabaja en las tripas del software''. \url{https://piranhalab.github.io/} (Consultado el \today)}, implementado en el marco de comisiones específicas, eventos institucionales e independientes. La premisa central de este proyecto fue escribir una serie de módulos de software para que los asistentes a eventos performáticos audiovisuales pudieran compartir una experiencia ligera para el navegador de manera co-presencial, aprovechando las posibilidades de las tecnologías de transmisión de audio y video. % Primero la hipotesis de trabajo  

La escritura de \textit{Panorama} se delimita a \textit{EDGES}\footnote{``Plataforma de experimentación y difusión de proyectos audiovisuales en vivo'' impulsada por el Laboratorio de Imágenes en Movimiento del Centro Multimedia del Centro Nacional de las Artes. \url{https://www.facebook.com/events/209679013466792} (Consultado el \today)}. Los conciertos realizados en este ciclo se llevaron a cabo del 6 de agosto al 19 de noviembre de 2020. Algunos eventos independientes que fueron significativos para la escritura sucedieron poco antes o después de estas fechas. 	

% Comenté lo siguiente, esto antes era parte de un resumen extenso, ahora parece redundante

% El artículo aporta elementos a una discusión para la reflexión transversal (técnica, estética y  de investigación académica) y utiliza conceptos que nos permiten desplazarnos entre estos hilos para entretejer la investigación transdisciplinaria. Se adscribe a los planteamientos de los estudios del software y busca extender la discusión del terreno técnico y descriptivo. A lo largo del texto buscamos problematizar el papel que juega la computadora (local o en servidores) en la realización de prácticas performáticas y audiovisuales en la web.

%% La discusión inicia con la idea del software como una prótesis y termina problematizando las ideas que orbitan en torno a la cajanegrización de cara a la idea del trabajo socialmente invertido para la escritura de software. % Esto puede ir en discusión 

% Pendiente: cómo resolvemos el problema de ser gestores, artistas y escritores. Investigación artística

% Segundo pendiente: realmente tenemos la capacidad narrativa de plantear una navegación 

% \subsection*{Ecosistema} % Esto antes se llamaba antecedentes ivaciones paralelasun titulo más serio sería: activaciones paralelas

\textit{Panorama} se insertó en un ecosistema que tuvo en común el diseño de la presentación (frontend) y el acceso a datos (backend) de espacios digitales tridimensionales que transmitieron señales de audio y video a un espacio digital, esto es, enfocaron una parte del ejercicio creativo en la previsualización, realización y mantenimiento de los recintos. Además, los proyectos buscaron resolver la transmisión de eventos con herramientas libres, algunos se enfocaron en la programación, desplazándose del uso de herramientas con funcionalidades previamente delimitadas a la configuración de programas personalizados. Finalmente, los espacios que mencionamos se distinguen de herramientas como Zoom, Jitsi, Google Meet; no atienden a la comunicación mediada por la palabra sino por las consecuencias estéticas de flujos audiovisuales. Los puntos de coincidencia entre los espacios referenciados en este apartado fueron: audio y video transmitido en tiempo real y la posibilidad de posicionar pantallas, audio, avatares y escenarios en el espacio. La alusión a las plataformas que compartieron ecosistema con \textit{Panorama} se delimita proyectos con cercanía performática como es el caso de la comunidad que practica la programación al vuelo o \textit{live coding}.

% Quité la parte espacial. 

%%%%%%%%%%%%%%%%%%%%%%%%%
%%%%% Distinciones %%%%%%
%%%%%%%%%%%%%%%%%%%%%%%%%

%% Por aquí podría ir la cuestión de la crítica a la escala de recursos y al uso de plataformas a través de redes sociales y servicios convencionales. La problematización no se centra en lo tecnológico sino en la estructuración de plataformas y redes. 

% Primera distinción: escritura de espacio en contraposición a uso de herramientas privativas o libres.
% Segunda distinción: Herramientas que tienen un objetivo comunicativo a través de la palabra (tipo pedagógico) 

En este contexto, los Algoraves organizados por Algo:ritmi\footnote{\url{https://www.facebook.com/AlgoritmiTorino/about/} (Consultado el \today)} iniciaron el interés por los espacios tridimensionales de realidad virtual para lidiar con el distanciamiento social de la pandemia. Estos eventos tuvieron lugar en Mozilla Hubs\footnote{\url{https://hubs.mozilla.com/} (Consultado el \today)}. Esta herramienta resuelve el backend de la experiencia y permite al diseñador de espacio centrarse en el montaje del escenario al que acceden los usuarios. Para el caso de la comunidad creativa en México, algunos otros casos de implementación de espacios virtuales en situaciones de concierto fueron propuestas por TOPLAP México. De manera similar a Algo:ritmi, Algoraves eventos relacionados con la escena de la programación al vuelo\footnote{\url{https://networkmusicfestival.org/programme/performances/toplap-mexico-vr-algorave/} (Consultado el \today)} fueron organizados en FabricaVR, la plataforma de realidad virtual dedicada de TOPLAP México. Ambos casos forman parte de comunidades que antes de la pandemia, realizaban conciertos con tecnologías de transmisión de audio y video\footnote{\url{https://www.youtube.com/c/Eulerroom/videos}(Consultado el \today)}.% Nombres de las personas 

Algunos otros proyectos utilizaron otro tipo de entornos para resolver el diseño de escenarios y el control de datos. Tal fue el caso de la plataforma OXXXO de Carlos Pesina que en co-participación con el festival Ceremonia permitió la realización de eventos digitales para afrontar la cancelación de conciertos presenciales causada por la pandemia de COVID-19.\footnote{La documentación es escasa. Ver: \url{https://www.instagram.com/p/B_bdw_TlrKa} (Consultado el \today) y \url{https://medium.com/@desyfree/musicaenvivocovid19-79e570a8f321} (Consultado el \today)} \textit{A-frame}\footnote{\textit{A-frame} es ``un marco de trabajo para construir experiencias de realidad virtual(VR)''. \url{https://aframe.io/docs/1.2.0/introduction/} (Consultado el \today)} fue una solución alternativa a Mozilla Hubs para la realización de eventos virtuales. \textit{Sinestesia}\footnote{``Laboratorio de Experimentación, Improvisación y Nuevos Medios''. \url{https://www.instagram.com/si.nestes.ia/}(Consultado el \today)} y \textit{ZeYX Lab}\footnote{\url{https://zeyxlab.com/}(Consultado el \today)} implementaron este entorno. Three.js\footnote{``El proyecto de three.js apunta a la creación de una librería 3D fácil de usar, ligera, multinavegador, multipropósito''. \url{https://threejs.org/} (Consultado el \today)} fue otro \textit{framework} elegido para realizar este tipo de experiencias para el navegador. Sobre este punto también destacamos el caso de \textit{Calindros}\footnote{\url{https://calindros.site/} (Consultado el \today)} de Hugo Solís. Este marco de trabajo resuelve el render gráfico, el resto de los complementos (comunicación entre pares, transmisión y procesamiento de señales de audio y video) debe ser resuelto de manera independiente. % Links caídos. Extender. Preguntar sobre el papel del juego. Falta un sitio más. Preguntar  

\textit{Federacion-de-codigo-al-vuelo} \citep{en-vivo}, conformado por Rodrigo Frenk, Diego Villaseñor, Dorian Villlaseñor, Marianne Teixido y Emilio Ocelotl, fue un antecedente directo del proyecto. Este repositorio y las discusiones que lo articularon fueron parteaguas de algunas ideas centrales que desembocaron en \textit{Panorama} pero que también influyeron en Camposónico\footnote{\url{https://echoic.space/} (Consultado el \today)} \citep{camposonico}. La experiencia de colaboración con la \textit{federación} permitió concebir el entramado de módulos de código que pudieran resolver aspectos específicos. Tal es el caso de un chat o alguna forma de escritura en tiempo real o la transmisión de gestos de un avatar que extiendieran las posibilidades de desplazamiento y movimiento corporal en el espacio. La búsqueda por un espacio personalizado de los proyectos que se desprendieron de la \textit{federación} se estableció con el objetivo de resolver problemáticas de rendimiento que las plataformas de alto nivel resolvían de una manera poco transparente. 

% Esto lo comenté igual pueden hablar más de esto en plataformas

% Los eventos anteriormente enunciados resolvieron la transmisión de audio y video con plataformas privativas de streaming como YouTube y Twitch; plataformas para envío bidireccional de señales de audio como Sagora, Jacktrip y Sonobus y plataformas para el montaje de servidores de audio y video como Icecast o LiquidSoap. Esta última experiencia apuntó a la escritura de un servidor personalizado de streaming de audio y video. % falta hablar de másservicios privativos de streaming

De la experiencia como asistentes, como ejecutantes y como incipientes investigadores/escritores de código en plataformas y eventos que implicaron estas resoluciones fue que el proyecto de \textit{PiranhaLab} empezó a delimitarse. Adicionalmente, sugieron discusiones sobre estos entornos, de las piezas y eventos que contuvieron e incluso del momento en el que espacio y e interpretación se desdibujaron. 
