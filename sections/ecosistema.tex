
% \subsection*{Ecosistema} % Esto antes se llamaba antecedentes ivaciones paralelasun titulo más serio sería: activaciones paralelas

El siguiente apartado describe una serie de plataformas distintas a \textit{Panorama} que permitieran realizar conciertos audiovisuales en el navegador con entornos tridimensionales. Los puntos de partida que detectamos son: audio y video transmitido en tiempo real y la posibilidad de posicionar pantallas, audio, avatares y escenarios en el espacio. La alusión a las plataformas que compartieron ecosistema con \textit{Panorama} se delimita proyectos con cercanía performática, la mayoría de ellos escritos e implementados en la Ciudad de México. 

En el contexto de la programación al vuelo o \textit{live coding}, los Algoraves organizados por Algo:ritmi\footnote{\url{https://www.facebook.com/AlgoritmiTorino/about/}} iniciaron el interés por los espacios tridimensionales de realidad virtual para lidiar con el distanciamiento social de la pandemia. Estos eventos tuvieron lugar en Mozilla Hubs\footnote{\url{https://hubs.mozilla.com/}}. Esta plataforma resuelve el backend de la experiencia y permite al diseñador de espacio centrarse en el frontend, el montaje del escenario al que acceden los usuarios.

% El diseño de espacios de arquitecturas colosales fue un rasgo distintivo de estos eventos. % esto suena medio chistoso 

Para el caso de la comunidad creativa en México, algunos otros casos de implementación de espacios virtuales en situaciones de concierto fueron propuestas por TOPLAP México. De manera similar a Algo:ritmi, Algoraves eventos relacionados con la escena de la programación al vuelo\footnote{TOPLAP México: VR Algorave consultado en: \url{https://networkmusicfestival.org/programme/performances/toplap-mexico-vr-algorave/}} fueron organizados en FabricaVR, la plataforma de realidad virtual dedicada de TOPLAP México. Ambos casos forman parte de comunidades que antes de la pandemia, realizaban conciertos con tecnologías de transmisión de audio y video\footnote{Véase: \url{https://www.youtube.com/c/Eulerroom/videos}}.% Nombres de las personas 

Adicionalmente a estos eventos, otras plataformas plantearon eventos similares con otros módulos de software. Tal fue el caso de la plataforma OXXXO de Carlos Pesina que en co-participación con el festival Ceremonia permitió la realización de eventos digitales para afrontar la cancelación de conciertos presenciales.\footnote{La documentación es escasa. Ver: \url{https://www.instagram.com/p/B_bdw_TlrKa}, también: \url{https://medium.com/@desyfree/musicaenvivocovid19-79e570a8f321}}

A-frame\footnote{ A-frame es ``un marco de trabajo para construir experiencias de realidad virtual(VR)''. Consultado en: \url{https://aframe.io/docs/1.2.0/introduction/}} fue una solución alternativa a Mozilla Hubs para la realización de eventos virtuales. \textit{Sinestesia}\footnote{``Laboratorio de Experimentación, Improvisación y Nuevos Medios''. Consultado en: \url{https://www.instagram.com/si.nestes.ia/}} y \textit{ZeYX Lab}\footnote{\url{https://zeyxlab.com/}} implementaron este entorno. Three.js\footnote{``El proyecto de three.js apunta a la creación de una librería 3D fácil de usar, ligera, multinavegador, multipropósito''. \url{https://threejs.org/}} fue otro \textit{framework} elegido para realizar este tipo de experiencias para el navegador. En este sentido destacamos el caso de \textit{Calindros}\footnote{\url{https://calindros.site/}} de Hugo Solís. Este marco de trabajo resuelve el render gráfico y el resto de los complementos debe ser resuelto de manera independiente. % Links caídos. Extender. Preguntar sobre el papel del juego. Falta un sitio más. Preguntar  

Finalmente, \textit{federacion-de-codigo-al-vuelo}\footnote{\url{https://github.com/federacion-de-codigo-al-vuelo}} fue el parteaguas de algunas ideas centrales que desembocaron en \textit{Panorama} pero también en otros dos proyectos manejados por Rodrigo Frenk y Diego Villaseñor respectivamente: Zona hipermedial y Camposónico\footnote{\url{https://github.com/diegovdc/camposonico}}. La experiencia de colaboración con la \textit{federación} permitió sentar las bases el entramado de módulos de código que pudieran resolver aspectos específicos. Por ejemplo,  un chat o alguna forma de escritura en tiempo real, la transmisión de gestos de un avatar que extiendieran las posibilidades de desplazamiento y movimiento corporal en el espacio. La búsqueda por un espacio personalizado de los proyectos que se desprendieron de la \textit{federación} se estableció con el objetivo de resolver las problemáticas de rendimiento que las plataformas de alto nivel resolvían de una manera poco transparente. 

Los eventos anteriormente enunciados resolvieron la transmisión de audio y video con  plataformas privativas de streaming como YouTube y Twitch; plataformas para envío bidireccional de señales de audio como Sagora, jacktrip y Sonobus y plataformas para el montaje de servidores de audio y video como Icecast. Esta experiencia apuntó a la escritura de un servidor personalizado de streaming de audio y video. % falta hablar de másservicios privativos de streaming

De la experiencia como asistentes, como ejecutantes y como incipientes investigadores/escritores de código en plataformas y eventos que implicaron estas resoluciones fue que el proyecto de \textit{PiranhaLab} empezó a delimitarse. Adicionalmente, sugieron algunas cuestiones referentes a la observación e investigación de estos entornos, de las piezas que contienen e incluso del momento en el que espacio y pieza se desdibujan. 
