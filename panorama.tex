% Panorama

% Un cuarto propio conectado. Remedios Zafra > Pdf revisar para referencias 

\documentclass[12pt,letterpaper, twocolumn]{article}
\usepackage[utf8]{inputenc}
\usepackage[spanish]{babel}
\usepackage{endnotes}
\usepackage{hyperref}
\usepackage{amsmath}
\usepackage{amsfonts}
\usepackage{amssymb}
\usepackage[sort&compress]{natbib}

\renewcommand{\notesname}{Notas}

\hypersetup{
    colorlinks=true,
    linkcolor=blue,
    filecolor=magenta,      
    urlcolor=magenta,
    citecolor=magenta,
}

\urlstyle{same}

\author{Marianne Teixido, Dorian Sotomayor, Emilio Ocelotl}
%\title{Panorama. Tecnologías libres e inmersivas para el performance audiovisual}

\title{%
  Panorama \\
  \large Escritura de espacios libres e inmersivos para el performance audiovisual}

\providecommand{\keywords}[1]
{
  \small	
  \textbf{\textit{Palabras clave---}} #1
}

\let\footnote=\endnote

\begin{document}

\maketitle

\begin{abstract}
  
El confinamiento provocado por la pandemia de COVID-19 obligó a artistas, gestores, instituciones públicas e industrias a replantear maneras de compartir flujos co-presenciales y hacer performance audiovisual en vivo. La presente investigación se enmarca en está búsqueda; el distanciamiento social forzado fue el pretexto para resolver necesidades tecnológicas que pudieran equilibrar la perspectiva funcional y la experimental. Este trayecto desembocó en discusiones sobre materialidad y virtualidad.                          

El presente artículo describe \textit{Panorama}\footnote{Repositorio de Panorama.}, un conjunto de módulos de código y software que permiten realizar conciertos en espacios virtuales, tridimensionales alojados en la web. La hipótesis/premisa central de este proyecto buscó que los usuarios pudieran compartir una experiencia ligera para el navegador de manera co-presencial, aprovechando las posibilidades de las tecnologías de transmisión de audio y video.  

\end{abstract}


\keywords{Transmisión, streaming, multiplayer, 3d, espacio, audio, video, materialidad, virtualidad, cyborg}




\section*{Introducción y antecedentes} % Antes ecosistema 

\textit{Panorama} \citep{panorama} fue un programa escrito por \textit{PiranhaLab}\footnote{``PiranhaLab es un laboratorio interdisciplinario que trabaja en las tripas del software''. \url{https://piranhalab.github.io/} (Consultado el \today)}, implementado en el marco de conciertos realizados en la web. La premisa central de este proyecto fue escribir una serie de módulos de software para que los asistentes a eventos performáticos audiovisuales pudieran compartir una experiencia ligera para el navegador de manera co-presencial, aprovechando las posibilidades de las tecnologías de transmisión de audio y video. % Primero la hipotesis de trabajo  

La presente investigación hace referencia a casos del ciclo de conciertos \textit{EDGES}, plataforma de experimentación y difusión de proyectos audiovisuales en vivo, impulsada por el Laboratorio de Imágenes en Movimiento del Centro Multimedia del Centro Nacional de las Artes.\footnote{\url{https://www.facebook.com/events/209679013466792} (Consultado el \today)} Los conciertos realizados en este ciclo se llevaron a cabo del 6 de agosto al 19 de noviembre de 2020.

% Algunos eventos independientes que fueron significativos para la escritura sucedieron poco antes o después de estas fechas.

% Comenté lo siguiente, esto antes era parte de un resumen extenso, ahora parece redundante

% El artículo aporta elementos a una discusión para la reflexión transversal (técnica, estética y  de investigación académica) y utiliza conceptos que nos permiten desplazarnos entre estos hilos para entretejer la investigación transdisciplinaria. Se adscribe a los planteamientos de los estudios del software y busca extender la discusión del terreno técnico y descriptivo. A lo largo del texto buscamos problematizar el papel que juega la computadora (local o en servidores) en la realización de prácticas performáticas y audiovisuales en la web.

%% La discusión inicia con la idea del software como una prótesis y termina problematizando las ideas que orbitan en torno a la cajanegrización de cara a la idea del trabajo socialmente invertido para la escritura de software. % Esto puede ir en discusión 

% Pendiente: cómo resolvemos el problema de ser gestores, artistas y escritores. Investigación artística

% Segundo pendiente: realmente tenemos la capacidad narrativa de plantear una navegación 

% \subsection*{Ecosistema} % Esto antes se llamaba antecedentes ivaciones paralelasun titulo más serio sería: activaciones paralelas

\textit{Panorama} se insertó en un ecosistema que tuvo en común el diseño de la presentación al usuario (frontend) y el acceso a datos (backend) de espacios digitales tridimensionales que transmitieron señales de audio y video a un espacio digital, esto es, enfocaron una parte del ejercicio creativo en la previsualización, realización y mantenimiento de los recintos que fueron accedidos por medio de navegadores web.

Además, los proyectos buscaron resolver la transmisión de eventos con software libre, algunos se enfocaron en la programación, desplazándose del uso de herramientas desarrolladas por terceros  a la escritura programas personalizados.

Finalmente, los espacios que mencionamos se distinguen de herramientas como Zoom, Jitsi, Google Meet en lo que respecta al tipo de comunicación que procuran: los espacios se centraron en las consecuencias estéticas de flujos audiovisuales y no en la palabra. Los puntos de coincidencia entre los espacios referenciados en este apartado fueron: audio y video transmitido en tiempo real y la posibilidad de posicionar pantallas, audio, avatares y escenarios en el espacio. Los proyectos que realizaron eventos en la web y que compartieron ecosistema con \textit{Panorama} tienen una cercanía performática como es el caso de la comunidad que practica la programación al vuelo o \textit{live coding}.

% Quité la parte espacial. 

%%%%%%%%%%%%%%%%%%%%%%%%%
%%%%% Distinciones %%%%%%
%%%%%%%%%%%%%%%%%%%%%%%%%

%% Por aquí podría ir la cuestión de la crítica a la escala de recursos y al uso de plataformas a través de redes sociales y servicios convencionales. La problematización no se centra en lo tecnológico sino en la estructuración de plataformas y redes. 

% Primera distinción: escritura de espacio en contraposición a uso de herramientas privativas o libres.
% Segunda distinción: Herramientas que tienen un objetivo comunicativo a través de la palabra (tipo pedagógico) 

En este contexto, los Algoraves organizados por Algo:ritmi\footnote{\url{https://www.facebook.com/AlgoritmiTorino/about/} (Consultado el \today)} iniciaron el interés por los espacios tridimensionales de realidad virtual para lidiar con el distanciamiento social de la pandemia. Estos eventos tuvieron lugar en Mozilla Hubs\footnote{\url{https://hubs.mozilla.com/} (Consultado el \today)}. Esta herramienta resuelve el backend de la experiencia y permite al diseñador de espacio centrarse en el montaje del escenario al que acceden los usuarios.

Algunos otros casos de implementación de espacios virtuales en situaciones de concierto fueron propuestas por ToplapMx. De manera similar a Algo:ritmi, Algoraves eventos relacionados con la escena de la programación al vuelo\footnote{\url{https://networkmusicfestival.org/programme/performances/toplap-mexico-vr-algorave/} (Consultado el \today)} fueron organizados en FabricaVR, la plataforma de realidad virtual dedicada de TOPLAP México. Ambos casos forman parte de comunidades que antes de la pandemia, realizaban conciertos con tecnologías de transmisión de audio y video\footnote{\url{https://www.youtube.com/c/Eulerroom/videos}(Consultado el \today)}.% Nombres de las personas 

Algunos otros proyectos utilizaron otro tipo de entornos para resolver el diseño de escenarios y el control de datos. Tal fue el caso de la plataforma OXXXO de Carlos Pesina que en co-participación con el festival Ceremonia permitió la realización de eventos digitales para afrontar la cancelación de conciertos presenciales causada por la pandemia de COVID-19.\footnote{La documentación es escasa. Ver: \url{https://www.instagram.com/p/B_bdw_TlrKa} (Consultado el \today) y \url{https://medium.com/@desyfree/musicaenvivocovid19-79e570a8f321} (Consultado el \today)} \textit{A-frame}\footnote{\textit{A-frame} es ``un marco de trabajo para construir experiencias de realidad virtual(VR)''. \url{https://aframe.io/docs/1.2.0/introduction/} (Consultado el \today)} fue una solución alternativa a Mozilla Hubs para la realización de eventos virtuales. \textit{Sinestesia}\footnote{``Laboratorio de Experimentación, Improvisación y Nuevos Medios''. \url{https://www.instagram.com/si.nestes.ia/}(Consultado el \today)} y \textit{ZeYX Lab}\footnote{\url{https://zeyxlab.com/}(Consultado el \today)} implementaron este entorno.

Three.js\footnote{``El proyecto de three.js apunta a la creación de una librería 3D fácil de usar, ligera, multinavegador, multipropósito''. \url{https://threejs.org/} (Consultado el \today)} fue otro \textit{framework} elegido para realizar este tipo de experiencias para el navegador. Sobre este punto también destacamos el caso de \textit{Calindros}\footnote{\url{https://calindros.site/} (Consultado el \today)} de Hugo Solís. Este marco de trabajo resuelve el render gráfico, el resto de los complementos (comunicación entre pares, transmisión y procesamiento de señales de audio y video) debe ser resuelto de manera independiente. % Links caídos. Extender. Preguntar sobre el papel del juego. Falta un sitio más. Preguntar  

\textit{Federacion-de-codigo-al-vuelo} \citep{en-vivo}, conformado por Rodrigo Frenk, Diego Villaseñor, Dorian Sotomayor, Marianne Teixido y Emilio Ocelotl, fue un antecedente directo del proyecto. Este repositorio y las discusiones que lo articularon fueron parteaguas de algunas ideas centrales que desembocaron en \textit{Panorama} pero que también influyeron en Camposónico\footnote{\url{https://echoic.space/} (Consultado el \today)} \citep{camposonico}. La experiencia de colaboración con la \textit{federación} permitió concebir el entramado de módulos de código que pudieran resolver aspectos específicos. Tal es el caso de alguna forma de escritura en tiempo real o la transmisión de gestos de un avatar que extendiera las posibilidades de desplazamiento y movimiento corporal en el espacio. La búsqueda por un espacio personalizado de los proyectos que se desprendieron de la \textit{federación} se estableció con el objetivo de resolver problemáticas de rendimiento que las plataformas de alto nivel resolvían de una manera poco transparente. 

% Esto lo comenté igual pueden hablar más de esto en plataformas

% Los eventos anteriormente enunciados resolvieron la transmisión de audio y video con plataformas privativas de streaming como YouTube y Twitch; plataformas para envío bidireccional de señales de audio como Sagora, Jacktrip y Sonobus y plataformas para el montaje de servidores de audio y video como Icecast o LiquidSoap. Esta última experiencia apuntó a la escritura de un servidor personalizado de streaming de audio y video. % falta hablar de másservicios privativos de streaming

% De la experiencia como asistentes, como ejecutantes y como incipientes investigadores/escritores de código en plataformas y eventos que implicaron estas resoluciones fue que el proyecto de \textit{PiranhaLab} empezó a delimitarse. Adicionalmente, sugieron discusiones sobre estos entornos, de las piezas y eventos que contuvieron e incluso del momento en el que espacio y e interpretación se desdibujaron. 
 % Introducción, antecedentes, proyectos y tecnologías similares.
% aclarar algunas cuestiones sobre los módulos para que no sean tan críptico 

\section*{Diseño y escritura} % Aquí puede ir lo que escribió Dorian

El diseño de \textit{Panorama} consideró:

\begin{itemize}
\item Escritura de una experiencia ligera: tiempo de carga, percepción de movimiento del punto de vista de la cámara y percepción de otros usuarios en un espacio tridimensional.  
\item Utilización sencilla, adaptable a distintos dispositivos y a la capacidad técnica de los usuarios.
\item Transmisión audiovisual en vivo que permitiera simular la experiencia de asistencia a conciertos para eventos recurrentes y no recurrentes. 
\item Autonomía en el uso de recursos: el sistema no implicó un registro y buscó evadir el análisis del \textit{streaming} o de la conexión entre módulos para evitar transmisiones silenciadas por el uso de materiales con derechos de autor. 
\item Sistemas fáciles de instalar, recrear y escalar.
\item Tecnologías normalizadas, no privativas, con una marcada preferencia por el software libre.
\item Generación de experiencias personalizadas que respondiera a las necesidades performativas, posibilidades futuras abiertas y adaptables. 
\end{itemize}

\textit{Panorama} se escribió para entornos web, no requiere instalación de software adicional, más allá de un navegador web actual. Los avances en la tecnología web permiten la homogeneización de la experiencia y pueden abarcar un gran público. El sistema utiliza WebGL\footnote{\url{https://www.khronos.org/webgl/}(Consultado el \today)} para la renderización de imagen en el navegador. Este entorno permite la generación de espacios virtuales y tridimensionales donde se pueden alojar experiencias visuales, auditivas y presenciales. Javascript es el lenguaje de programación principal del proyecto debido a su uso normalizado en navegadores y aplicaciones web. Para fines prácticos, este sistema permite conectar las partes del sistema entero.

Three.js como \textit{framework} implementa webGL y asegura la rápida codificación y reutilización de código. Además, es posible importar modelos hechos en programas de modelado tridimensional, efectos y módulos para la reproducción de audio en web con Web Audio API\footnote{``La API de Audio Web provee un sistema poderoso y versatil para controlar audio en la Web, permitiendo a los desarrolladores escoger fuentes de audio, agregar efectos al audio, crear visualizaciones de audios, aplicar efectos espaciales (como paneo) y mucho más.'' \url{https://developer.mozilla.org/es/docs/Web/API/Web_Audio_API} (Consultado el \today)} y conceptos convencionales y compartidos de materiales y geometrías relacionados a programas de modelado en tes dimensiones. Como un punto adicional, este marco de trabajo cuenta con documentación actualizada y constante. La implementación de WebAssembly\footnote{``WebAssembly es un nuevo tipo de código que puede ser ejecutado en navegadores modernos — es un lenguaje de bajo nivel, similar al lenguaje ensamblador, con un formato binario compacto". \url{https://developer.mozilla.org/es/docs/WebAssembly} (Consultado el \today) } ha permitido la generación de sistemas rápidos y eficientes, sin capas de abstracción de software innecesarias.

El diseño del módulo de \textit{streaming} presentó una discusión sobre lo legal, la reproducción de audio y video en la web y uso de recursos. Las restricciones que imponen los servidores privados al contenido remixeado fue una de las motivaciones para implementar un sistema personalizado. De esta manera, el sistema evita el mercado exclusivo de servicios de streaming privados, con licencias de paga o limitados. De manera complementaria, el sistema de \textit{streaming} personalizado permite controlar la cantidad de recursos usados y usuarios conectados. Los formatos para la transmisión de video en web explorados fueron: RTMP (Real Time Messaging Protocol), FLV (Flash Video), HLS (HTTP Live Streaming), video a través de WebSockets\footnote{``WebSockets es una tecnología avanzada que hace posible abrir una sesión de comunicación interactiva entre el navegador del usuario y un servidor.'' \url{https://developer.mozilla.org/es/docs/Web/API/WebSockets_API}(Consultado el \today)}, WebRTC (Web Real-Time Communication) y MPEG-DASH (Dynamic Adaptive Streaming over HTTP).

Las características actuales presentes en HTML5 (Quinta versión de HyperText Markup Language) permiten utilizar formatos de streaming no nativos con \textit{frameworks} de decodificación para el despliegue de video en formatos soportados por el navegador. \textit{Panorama} utiliza RTMP debido a 1) la robustez del protocolo, 2) la rapidez de transmisión de datos, 3) la documentación que existe sobre NGINX\footnote{``Nginx es un servidor web/proxy inverso ligero de alto rendimiento". \url{https://es.wikipedia.org/wiki/Nginx}(Consultado el \today)} RTMP y 4) el uso de FFmpeg\footnote{``FFmpeg es el marco de trabajo multimedia líder capaz de decodificar, codificar, transcodificar, mux, demux, transmitir, filtrar y reproducir casi cualquier cosa que los humanos o las máquinas hayan creado". \url{https://www.ffmpeg.org/about.html}(Consultado el \today)} para la recodificación y redimensión de video en la web. La plataforma utiliza FLV debido a la velocidad del protocolo (no se descarta el uso de otros protocolos en el futuro). Para el uso de FLV en el navegador, se utilizó FLV.js. 

Para la interacción entre usuarios se utilizó WebSockets. Esto implicó: 1) uso de chat, 2) compartición en tiempo real de eventos gestuales (rotación y posición) del avatar utilizado y 3) personalización de modelo, textura y nombre de cada usuario. 

Del lado del servidor  se usaron balanceadores de carga junto con modelos de configuración para la instalación y eliminación de recursos, haciendo independientes los servidores dedicados al \textit{streaming} y a servicios web. Esto permite que los espacios puedan ser escritos, eliminados o mantenidos según el diseño del evento en cuestión.

% Escribir sobre las peculiaridades del git que estamos referenciando 

 % Elementos del diseño de Panorama. Enunciación y descripción breve de las tecnologías implicadas.

\section*{Notas y Pruebas} % Espacio para hablar del proyecto de maestría de Marianne, 

%% Escribir lo que sigue de alguna otra manera. 

\textit{Notas de Ausencia} \citep{notasdeausencia} es un ensayo generativo en la web. Utiliza el texto dato que por medio de la computadora como agente resignificante, deconstruye estructuras discursivas para resemantizar la narrativa sobre las desapariciones de mujeres en México y América Latina.

El tiempo y espacio virtual conforman una partitura para la memoria y la denuncia. La narrativa, semi autónoma, argumenta a partir de textos tomados de tweets, poemarios, libros y artículos feministas que explican desde la teoría las desapariciones forzadas, el feminicidio y la violencia de género. Lo cuales están presente como texto, imagen y sonido en un espacio tridemencional diseñado para funcionar a manera de memorial.

La narrativa de la pieza está articulada mediante la intervención de dos bots. El primero comparte tuits que localiza con hashtag como: \#MéxicoFeminicida, \#MadresEnBúsqueda, \#ViolenciadeGenero, \#NiUnaMenos, entre otros.\footnote{\url{https://twitter.com/notasausencia} (Consultado el \today)} Así como con un bot de generación de texto automático que remixea los textos selecionados por medio de cadenas de Markov.

\begin{figure}[H]
  \includegraphics[width=\textwidth]{img/notas.png}
  \caption{Notas de Ausencia de Marianne Teixido}
\end{figure}

Esta pieza se realizó en el contexto de la exhibición en línea Creaciones con algoritmos: visualización y sonificación de datos del Centro de Cultura Digital en abril de 2020. Su salida oficial se realizó en video sin embargo, el planeamiento original contempló la creación de un espacio virtual en la web que pudiera visualizar en tiempo real la información viva proveniente de los tweets. 
En respuesta a la necesidad de ubicar la pieza en una espacio tridimencional en la web, en colaboración con \textit{PiranhaLab} se planeó el uso de Three.js como solución para conectar dicho espacio y con la información proporcionada por los bots. 

A raíz de dicho proceso artístico generamos las condiciones técnicas referentes a la relativa autonomía del servidor requerido para mantener la pieza en línea, la cual permanece activa.

Por otro lado \textit{Pruebas Proféticas} fue el evento piloto que implementó por primera vez dos tipos de tecnologías específicas: exploración multijugador y \textit{streaming} personalizado, montados en un escenario tridimensional previamente explorado en Notas de Ausencia. 

\begin{figure}[H]
  \includegraphics[width=\textwidth]{img/pruebasprofeticas.jpg}
  \caption{Pruebas proféticas. Visuales: Flor de Fuego}
\end{figure}

%La transmisión de audio y video fue un aspecto que el planteamiento de \textit{PiranhaLab} buscó solucionar. Es posible utilizar servicios gratuitos o de paga para la transmisión de datos audiovisuales, sin embargo, en menor o mayor medida, el flujo audiovisual generado es analizado y en caso de que se detecte algún extacto de audio proveniente con derechos de autor, el stream es silenciado. Este artículo no busca centrarse en discusiones sobre derechos de autor sino en la usabilidad de un streaming audiovisual. 


 \color{Fuchsia}

La experiencia con \textit{Pruebas Proféticas} abrió camino para el diseño de la edición 2020 de EDGES. Durante el proceso el proyecto nos permitió reflexionar en torno a funcionalidad y experimentación como dos posibilidades de un continuo para la escritura de software en un marco artístico y performático. EDGES como plataforma explicita el papel experimental de los actos, la plataforma tecnológica también podría ser experimental e incluso podría desdibujarse en pos de la integración performance-espacio bajo la misma premisa de la experimentación.

\color{black}

El concepto curatorial de \textit{EDGES} estuvo definido por Marianne Teixido y guardó una estrecha relación con los planteamientos de \textit{Notas de Ausencia}, considera las posibilidades de creación planteandonos desde el feminismo intereseccional, perspectiva desde la cual se problematiza el uso de la tecnología teniendo en consideración las condiciones de raza, género y clase, a partir de la cuales se establece una crítica las herramientas hegemónicas ya dadas para apuntar a la creación de estas otras herramientas, construídas desde dinámicas de organización colectiva y conocimientos situados. \textit{EDGES 2020} como propuesta curatorial contempla la participación en su mayoría de mujeres y persona no binaries. Las obras dialogan con las hibridaciones e-corporales en espacios virtuales ficcionados desde las subjetividades feministas y transfeministas que toman internet como territorio y espacio de intercambio cultural.

\iffalse
- Uso de espacios tridimensionales 
- Bots y literatura 
- Datos que transforman el espacio   
- Ensayos digitales en la web
- cyberfeminisimo
- audio virtualmente posicionado 
- streaming de audio y video sin plataformas privativas - decisiones de optimización
- Según yo aquí usamos icecast y liquid soap 
- Inicios de multiplayer
\fi
 % Primera parte que describe resultados. Casos elegidos: Notas de Ausencia y Pruebas proféticas

\section*{Contemplación y EDGES} % Espacio para hablar del proyecto de (acomplete) de Dorian

\iffalse
\begin{itemize}
\item Distopía
\item domo 
\item Underborders
\item Milena y Concepción
\item setInterval()
\item Contemplación del fin del Mundo
\item sistemas mixtos
\item espacio y performance fusionados en Contemplación
\end{itemize}
\fi

\textit{Distopía}, \textit{NLXS + NK}, \textit{Interconexión}, \textit{setInterval()} y \textit{La Contemplación del Fin del Mundo} fueron los eventos realiados en el marco de \textit{EDGES}. 

\begin{figure}[H]
  \includegraphics[width=\textwidth]{img/distopia.png}
  \caption{EDGES Distopía. Imágenes: LVSTVCRV}
\end{figure}

\textit{La Contemplación del Fin del Mundo} es un performance a modo de ejuego, último evento de la serie \textit{EDGES} que destruye el escenario de manera simbólica para dar por finalizado el ciclo de conciertos. Los asistentes podían presenciar el fin del mundo con la destrucción del escenario y otros eventos como inundaciones, objetos celestiales y finalmente la dispersión de los colores del escenario, dejando a los objetos del espacio sin razgos reconocibles.

La idea principal sirvió como vehículo para la exploración del espacio como característica del performance, el mundo explorable, la persecución en forma de figuras celestiales que ocupaban todo el espacio o que se expandían e iluminaban todo así como las inundaciociones. También permitió la exploración del uso de pantallas distribuidas a lo largo de todo el mundo, permitiendo a los usuarios presenciar el performance desde cualquier ubicación.

Uno de los aspectos a destacar de este concierto es el uso de acciones colectivas lanzadas por el artista, que durante el trascurso del evento podía cambiar las características del ambiente de manera similar entre los participantes. La experiencia de los usuarios fue se transformaba en el trancurso del evento, fue homogénea y compartida. Adicionalmente se implementó Hydra \citep{hydra} como un framework externo para la creación de visuales.

Las dificultad de las experiencias compartidas radica en la sincronización de eventos, tanto para los usuarios que ingresan desde el inicio o los usuarios ocasionales, sin importar ubicación geográfica o dispositivo. Esta posibilidad permite la interacción del artista y genera situaciones que añadan dinámica al juego, donde los asistentes se desplazan de ser observadores a ser participantes activos.

% Escribir sobre el proyecto de vincular threejs y hydra 
  % Segunda parte que describe resultados. Casos elegidos: EDGES y La contemplación del Fin del Mundo

\section*{Three y 4NT1} % Espacio para hablar del proyecto de doctorado de Emilio

\textit{THREE.studies} \citep{threestudies} hereda discusiones referentes al punto de vista, la co-presencia, el envío de información gestual a través de la web, la transmisión de flujos de audio y video a partir de servidores y el uso de fuentes sonoras en un espacio virtual. Se relaciona con \textit{4NT1} \citep{anti} y \textit{tres-estudios-abiertos} \citep{tresestudios} y forma parte de un proyecto de investigación doctoral que aborda nuevas prácticas artísticas audiovisuales en el navegador a partir de lenguajes de programación.  

\begin{figure}[H]
\includegraphics[width=\textwidth]{img/three.png}
\end{figure}

La primera instancia de \textit{THREE.studies}, \textit{threecln}, es un performance audiovisual para el navegador. Las señales de audio y video se encuentran en un espacio diseñado para el evento. Los elementos del escenario interactúan con las señales y proveen de retroalimentación sonora y visual al intérprete musical.

El espacio se fusiona con la interpretación y resulta en una pieza para el navegador / partitura gráfica que se transforma a sí misma cada vez que se interpreta. La obra involucra a un intérprete musical, para el caso que revisamos en este artículo, de violonchelo eléctrico, el operador de la electrónica en vivo y el equipo que mantiene la estabilidad del espacio.

El intérprete musical envía un \textit{stream} que es espacializado y que interactúa con los elementos visuales de la escena. El resultado es una obra / partitura que puede explorarse en tiempo real por el público. 

Por otro lado, \textit{4NT1} busca problematizar las relaciones que existen entre usuarios y plataformas tecnológicas; es un paso hacia la realización de usuarixs que desdibujan las fronteras de la pasividad política y económica teniendo como epicentro lo sensible. El proyecto parte de la composición visual conducida por datos. Aprovecha la investigación y el desarrollo de tres estudios abiertos, un proyecto doctoral sobre nuevas prácticas artísticas en el navegador y librerías de síntesis granular para audio y video.

La obra toma en cuenta la transformación de flujos de audio y video y se retroalimenta con la acción de agentes externos. Con técnicas de aprendizaje automático, detecta gestos faciales que son intepretados como un flujo de datos. El proyecto problematiza este flujo con el uso de tecnologías que implican una responsabilidad de los datos de usuarixs. De esta manera el proyecto pplantea una discusión que parte de la instagramización de la política y la estetización de la resistencia para desembocar en la política de la representación.

\textit{4NT1} es un pedazo de software que puede utilizarse en la vida cotidiana y que desplaza la ofuscación en el uso de tecnologías que funcionan como cajas negras al desarrollo de capas estéticas para la evasión. El proyecto contempla la comparación de dos caminos que permitan plantear una crítica al software como caja negra. Es un primer estudio de reflexión tecno-social. Retoma la idea de modularidad y se adscribe a los estudios del software, esto quiere decir que la obra se complementa con la programación, lectura, escritura y pensamiento con software.

\iffalse

- Videotitlán ?
- THREE.studies
- ¿anti? 
\fi
 % Tercera parte que describe resultados. Casos elegidos: threecln (en proceso) y 4nt1 (en proceso)

\section*{Discusión}

Propuesta para esta sección: Desmantelar e integrar en resultados, descripción y conclusiones

\begin{itemize}

  \color{black}

\item {Black - Dejar en conclusiones}

  \color{BlueViolet}

\item BlueViolet - Para Contemplación 
  
  \color{Fuchsia}
  
\item Fuschia - Para Notas de Ausencia

  \color{MidnightBlue}

\item MidnightBlue - Para Anti

  \color{BlueGreen}

\item BlueGreen - Para Descripción tecnológica
  
\end{itemize}

% Aquí puede ir lo de los espacios y arquitecturas. También aquí puede ir lo de materialidad y neomaterialidad 

% Referencias a la consecuencia y no al objeto por sí mismo 

% La discusión en torno a la escritura de software inicia con la perspectiva prostética y termina problematizando las ideas que orbitan en torno a la cajanegrización de cara a la idea del trabajo socialmente invertido en esta actividad.

\color{BlueViolet}

%El término cyberespacio\footnote{``El término cyberespacio puede ser usado vagamente para describir la interconexión espacial e invisible de computadoras en el Internet y es aplicado a casi cualquier experiencia espacial y virtual creada en una computadora'' \citep{cyberspace}} puede ser un punto de partida para describir la actividad relacionada con \textit{Panorama}. Puede ser contrapuesto con diversos conceptos que coinciden en la práctica, en la arquitectura de intercambio de información, en la denominación conceptual y en la convergencia de tecnologías expresamente mercantilizadas para la distribución de experiencias de Realidad Virtual (VR) como Oculus Rift. VR también puede ser un concepto inicial, socialmente convenido para describir la experiencia de \textit{Panorama} pero para evitar contradicciones en la conducción de una experiencia en el cyberespacio hacia exploraciones hacia dispositivos/interfaces para la interacción en estos espacios, el presente artículo decide retomar la noción de inmersividad. Consideramos que la relación entre cyberespacio como una noción narrativa y de inmersividad como una forma de resolución técnica de exploración de una cámara en un espacio tridimensional, con audio posicionado\footnote{``En \textit{Web Audio}, espacializaciones complejas en 3D son creadas utilizando PannerNode, que en términos legos básicamente son muchas matemáticas \textit{cool} para hacer que el audio aparezca en un espacio 3D. Piensa en el audio volando sobre de ti, trepando atras de ti, moviéndose enfrente de ti" \url{https://developer.mozilla.org/en-US/docs/Web/API/Web_Audio_API/Web_audio_spatialization_basics} (Consultado el \today)}, puede arrojar algunas ideas sobre la actividad realizada en el marco de \textit{Panorama}. % Falta desarrollar esto 



\iffalse

- manifiestos y posturas políticas > live coding y servidora feminista 
- Diferencias con respecto a otros espacios
- Discusión sobre lo digital, los nuevos medios y la virtualidad. 
- Giro de los nuevos medios
- Agotamiento del formato
- funcionalidad - experimentación 
- Arqueología en internet (cosas que ya tienen tiempo y cosas recientes). 
- El papel de los videojuegos gamización
- Lo barroco y el artículo de electroacústica. 
- Espacio, arquitectura e integración: puede el espacio virtual convertirse en un objeto con agencia en la interpretacion en vivo? 

\fi

\color{black}
 % Como los resultados se vinculan con las reflexiones y perspectivas externas al artículo.

\section*{Conclusiones}

% La pregunta principial sería: ¿Se cumplió la hipótesis/premisa del software escrito? 

Los usuarios pudieron compartir una experiencia ligera para el navegador de manera co-presencial, aprovechando las posibilidades de las tecnologías de transmisión de audio y video. Destacamos la importancia de plantear soluciones compartibles en lo que respecta a la transmisión de audio-imagen. La escritura de esta parte de \textit{Panorama} fue utilizado en el marco de otros eventos y ciclos.

El proyecto no tuvo un plan de acción específico para la realización de mediciones durante los conciertos, a pesar de que el control del servidor lo permitió. Sin embargo, la lectura de información a partir de una simulación del evento arroja datos importantes en este sentido.

Sobre las impresiones del público, fue posible realizar un cuestionario durante \textit{Pruebas Proféticas}. 

La importancia de una infraestructura tecnológica y social. Esto se puede entretejer a partir de decisiones curatoriales que vinculan potencial tecnológico con alianzas entre actores. 

Del lado teórico es necesario convenir y reforzar un corpus de conceptos para la investigación sobre Tecnología Musical. Detección de campos y persectivas de investigación: las agencias que están presentes en eventos que se desenvuelven con tecnología.

Destacamos el giro de los nuevos medios como un cambio de paradigma que la tecnología musical debe considerar por las implicaciones de los usos/críticas de las tecnologías de la información, las tendencias para la resolución de problemas relacionados con gestión de datos y las consecuencias estéticas que a veces se empalman y otras rebasan a la música y las perspectivas de investigación asociadas a esta disciplina. 

El rodeo o la realización de un motivo tecnológico como una perspectiva de investigación que pueda aportar en el aspecto tecnológico y teórico-metodológico, sobre todo en campos que lo permiten como humanidades, artes y específicamente, investigación artística con tecnología.
 % ¿Se cumplió la premisa/hipótesis descrita en Diseño y Escritura? ¿Qué otras cosas surgieron?

\theendnotes % El artículo usa notas al final del texto. Aquí aparecen hipervínculos y conceptos de apoyo. Conceptos 

%\nocite{interactiveDigitalMusic}
% \nocite{speakingCode} 

\addcontentsline{toc}{chapter}{\protect\numberline{}Referencias}%
\bibliography{bib/panBib}{} % Referencias bibliográficas y repositorios en Git como referencias en la web.
\bibliographystyle{apalike-es}

\end{document}
